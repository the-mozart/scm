\documentclass{beamer}

\usepackage{beamerthemesplit}
\usetheme{Singapore} %Copenhagen}


\input{../../include/preamble.inc} 
\input{../../include/definitions.inc} 
\input{../../include/author.inc} 


\title[]{Линейные операторы в $n$-мерном пространстве}

\begin{document}
\frame[plain]{\titlepage}


\frame[plain]{
\frametitle{Аннотация}
\parbox{\textwidth}{
Линейные операторы в $n$-мерном пространстве. Подобные матрицы. Определитель оператора. Обратный оператор. Характеристические числа и собственные векторы линейного оператора. Две теоремы о собственных векторах. Линейные операторы простой структуры.
}
}

\frame{
\frametitle{Векторное пространство}

\begin{dfn}
%\rm
\parbox{\textwidth}{
Линейный оператор $\oper{A}:\Rn \to \Rn$, отображающий $n$-мерное векторное пространство  $\Rn$ само в себя, называется \alert{линейным оператором в $\Rn$}.
}
\end{dfn}


\medskip
\begin{exampleblock}{\only<2->{Особенности линейных операторов над $\Rn$}}
\begin{enumerate}

\item<3-> Cуществует оператор $\oper{E}$ (называемый единичным) такой, что  для любого вектора $\vec{x}$  из $R^n$ 
\[
	\oper{E}\vec{x}=\vec{x}.
\]

\item<4-> Для любого оператора $\oper{A}$ из $R^n$ справедливо соотношение
\[
	\oper{A}\oper{E}=\oper{E}\oper{A}=\oper{A}.
\]
\end{enumerate} 

\end{exampleblock}

}

\frame{
\frametitle{О матрице соответствующей линейному оператору в $\Rn$}

\parbox{\textwidth}{
Выберем базис в $\Rn$: $\vec{e_1}$, ..., $\vec{e_n}$. 
\pause В этом базисе оператору $\oper{A}$ соответствует квадратная $n\times n$ матрица $A$.
\pause Столбцы этой матрицы составлены из координат вектора $\oper{A}\vec{e_j}$ в базисе $\vec{e_i}$ $i,j=\overline{1,n}$.

\pause
\[
A =  \left(
\begin{array}{c|c|c|c}
\oper{A}\vec{e}_1 & \oper{A}\vec{e}_2 & \ldots & \oper{A}\vec{e}_n\\
\end{array}\right)=
\left(
\begin{array}{cccc}
a_{11} & a_{12} & \ldots & a_{1n} \\
a_{21} & a_{22} & \ldots & a_{2n} \\
\vdots & \vdots & \ddots & \vdots\\
a_{n1} & a_{n2} & \ldots & a_{nn} \\
\end{array}\right)
\]

}
}

\frame{
\frametitle{О действиях над линейными операторами в $\Rn$}

\begin{theorems}
\rm
\parbox{\textwidth}{
Пусть $\oper{A}$, $\oper{B}$, $\oper{C}$ -- линейные операторы в $\Rn$.
\pause
Пусть $A$, $B$, $C$ -- $n\times n$ матрицы соответствующие линейным операторам $\oper{A}$,
 $\oper{B}$, $\oper{C}$,
\pause а $x$ -- вектор столбец из координат вектора $\vec{x}$, 
 в  базисе $\vec{e_i}$ ($i=\overline{1,n}$) и $\alpha \in \mathbb{R}$, тогда 
\pause 
\begin{itemize}
\item   $\oper{C}=\oper{A}+\oper{B}\quad \Leftrightarrow \quad C=A+B$;
\pause 
\item   $\oper{C}=\oper{A}\oper{B}\quad \Leftrightarrow \quad C=AB$;
\pause 
\item   $\oper{C}=\alpha\oper{A}\quad \Leftrightarrow \quad C=\alpha A$.
\end{itemize}

(Доказательство вытекает из определения операций).
}
\end{theorems}

\begin{exampleblock}{Следствие}
\parbox{\textwidth}{
Множество линейных операторов линейно относительно сложения и умножения на число.
}
\end{exampleblock}
}

\frame{
\frametitle{О представлении единичного оператора в виде матрицы}

Пусть $\oper{E}$ -- единичный оператор в $\Rn$, \pause т.е. $\forall \vec{x}\in\Rn$ $\oper{E}\vec{x}=\vec{x}$.

\pause
Пусть $\vec{x}=x_1\vec{e}_1+x_2\vec{e}_2+\ldots+x_n\vec{e}_n$ \pause -- разложение вектора $\vec{x}$ по базису пространства $\Rn$ $\vec{e_i}$ $i=\overline{1,n}$.

\pause
Тогда 
\[
\begin{array}{lcl}
\oper{E}\vec{e}_1 &  =  & 1 \cdot \vec{e}_1 + 0\cdot\vec{e}_2+\ldots+0\cdot\vec{e}_n\\
\pause
\oper{E}\vec{e}_2 &  =  & 0 \cdot \vec{e}_1 + 1\cdot\vec{e}_2+\ldots+0\cdot\vec{e}_n\\
\vdots &  &  \\
\pause
\oper{E}\vec{e}_n &  =  & 0 \cdot \vec{e}_1 + 0\cdot\vec{e}_2+\ldots+1\cdot\vec{e}_n
\end{array}
\]

\pause
Следовательно оператору $\oper{E}$ соответствует единичная матрица:
\pause
\[
E=\left(
\begin{array}{cccc}
1 & 0 & \ldots & 0 \\
0 & 1 & \ldots & 0 \\
\vdots & \vdots & \ddots & \vdots\\
0 & 0 & \ldots & 1 \\
\end{array}\right).
\]


}

\frame{
\frametitle{О подобных матрицах}

\begin{dfn}
%\rm
\parbox{\textwidth}{
Две матрицы $A$ и $B$, \pause связанные соотношением $B=TAT^{-1}$, \pause называются \alert{подобными}, \pause где $T$ -- неособенная матрица.
}
\end{dfn}

\pause
\begin{theorems}
\rm
\parbox{\textwidth}{
Пусть $\oper{A}$ -- линейный оператор в $\Rn$, \pause
$A$ -- матрица, соответствующая оператору  $\oper{A}$ \pause в базисе  $\vec{e}_i$ ($i=\overline{1,n}$), \pause а $A'$ -- матрица, соответствующая оператору  $\oper{A}$ \pause в базисе  $\vec{g}_i$ ($i=\overline{1,n}$). \pause Тогда матрицы $A$ и $A'$ подобны. 
}
\end{theorems}




}

\frame{
\frametitle{О подобных матрицах}
\begin{proof}
\rm
\parbox{\textwidth}{
Пусть $T^{\rm{t}}$ -- неособенная матрица перехода \pause между базисами $\vec{e}_i$ и $\vec{g}_j$ ($i,j=\overline{1,n}$). \pause Пусть $\vec{y}=\oper{A}\vec{x}$ -- образ вектора $\vec{x}\in\Rn$. \pause
Пусть $x$ и $x'$ -- координаты вектора $\vec{x}\in\Rn$, \pause а $y$ и $y'$ -- координаты вектора $\vec{y}\in\Rn$ в заданных базисах. \pause Тогда
\[
y=Ax,\quad y'=A'x'.
\]

\pause
По теореме о связи векторов в различных базисах
\[
y'=Ty,\quad x'=Tx. 
\]

\pause
Подставляя выражение для $y'$ и $x'$ в предыдущие соотношения, получаем
\[
y=\left(T^{-1}A'T\right)x.
\]
\pause
Т.к. $y=Ax$, то в силу произвольности $\vec{x}$: \alert{$A=T^{-1}A'T$} .
}
\end{proof}
}

\frame{
\frametitle{Определитель линейного оператора в $\Rn$}

\parbox{\textwidth}{
Пусть $A$ и $B$ -- подобные матрицы. \pause Тогда $A=TBT^{-1}$ для некоторой неособой $T$ \pause и
\[
|A|=|TBT^{-1}|=|T||B||T^{-1}|=|T||B|/|T|=|B|.
\]
}

\pause
\begin{dfn}
%\rm
\parbox{\textwidth}{
\alert{Определителем} оператора $\oper{A}:R^n\rightarrow R^n$ \pause называется определитель матрицы преобразования этого оператора в любом базисе. \pause Если $|\oper{A}|=0$, то оператор называется \alert{особенным}, \pause в противном случае $|\oper{A}|\neq 0$ , оператор -- \alert{неособенный}.
}
\end{dfn}
}

\frame{
\frametitle{Свойства особенного оператора}

\begin{exampleblock}{Свойство}
\rm
\parbox{\textwidth}{
Для особенного оператора $\oper{A}$ \pause  существует такой вектор $\vec{x}\neq 0$ из $\Rn$, \pause что $\oper{A}\vec{x}=0$. \pause
}
\end{exampleblock}

\begin{proof}
\rm
\parbox{\textwidth}{
Выберем базис пространства $\Rn$ $\vec{e}_i$ ($i=\overline{1,n}$). \pause Пусть $A$ -- матрица оператора $\oper{A}$ в этом базисе. \pause Тогда операторное уравнение $\oper{A}\vec{x}=0$ \pause равнозначно матричному
\[
Ax=0.
\]

 \pause Оператор $\oper{A}$ особенный, значит $|A| = 0$.  \pause Следовательно матричное уравнение имеет нетривиальное решение $x\neq 0$.\\ \pause 
\centering
$\vec{x}=\sum\limits_{i=1}^n x_i \vec{e}_i \neq 0$ --- решение операторного уравнения.



}
\end{proof}

}

\frame{
\frametitle{Свойства неособенного оператора}

\begin{theorems}
\rm
\parbox{\textwidth}{

Неособенный оператор  $\oper{A}$ в $\Rn$ имеет следующие свойства:\pause
\begin{enumerate}
\item Из равенства $\oper{A}\vec{x}=0$ всегда следует, что $\vec{x}=0$.\pause
\item $\oper{A}\Rn=\Rn$ (полнота).\pause
\item $\oper{A}\vec{x}=\oper{A}\vec{y}\quad\Leftrightarrow\quad \vec{x}=\vec{y}$ (взаимнооднозначность).
\end{enumerate}

}
\end{theorems}

}

\frame{
\frametitle{1. $\oper{A}\vec{x}=0\quad \Rightarrow \quad\vec{x}=0$}

\begin{proof}
\rm
\parbox{\textwidth}{
Выберем базис пространства $\Rn$ $\vec{e}_i$ ($i=\overline{1,n}$).  \pause Пусть $A$ -- матрица оператора $\oper{A}$ в этом базисе. \pause  Тогда операторное уравнение $\oper{A}\vec{x}=0$ равнозначно матричному
\[
Ax=0.
\] \pause 
Оператор $\oper{A}$ неособенный, значит $|A|\neq 0$.  \pause Следовательно матричное уравнение имеет только одно решение $x=0$ и, следовательно, $\vec{x}=0$.

}
\end{proof}
}

\frame{
\frametitle{2. $\oper{A}\Rn=\Rn$ (полнота)}

\begin{proof}
\rm
\parbox{\textwidth}{
Нужно показать, что областью определения линейного неособенного оператора является всё $\Rn$. \pause 
Выберем базис пространства $\Rn$ $\vec{e}_i$ ($i=\overline{1,n}$).  \pause Пусть $A$ -- матрица оператора $\oper{A}$ в этом базисе.  \pause Нужно показать, что операторное уравнение $\oper{A}\vec{x}=\vec{y}$ имеет решение для любого вектора $\vec{y}\in\Rn$.  \pause Пусть $x$, $y$ -- вектор-столбцы координат векторов $\vec{x}$, $\vec{y}$ в выбранном базисе соответственно.  \pause В матричном виде уравнение $\oper{A}\vec{x}=\vec{y}$ имеет вид
\[
Ax=y.
\] \pause 
Оператор $\oper{A}$ неособенный, значит $|A|\neq 0$.  \pause Следовательно матричное уравнение имеет решение вида $x=A^{-1}y$. \\ \pause
\centering
$\vec{x}=\sum\limits_{i=1}^n x_i \vec{e}_i \neq 0$ --- решение операторного уравнения.
}
\end{proof}
}

\frame{
\frametitle{3. $\oper{A}\vec{x}=\oper{A}\vec{y}\quad\Leftrightarrow\quad \vec{x}=\vec{y}$ (взаимнооднозначность)}

\begin{proof}
\rm
\parbox{\textwidth}{
Выберем базис пространства $\Rn$ $\vec{e}_i$ ($i=\overline{1,n}$).  \pause Пусть $A$ -- матрица оператора $\oper{A}$ в этом базисе.  \pause 
\begin{itemize}
\item Пусть  $\oper{A}\vec{x}=\oper{A}\vec{y}$, т.е.  $\oper{A}(\vec{x}-\vec{y})=0$.  \pause Тогда из пункта 1 для неособенного оператора следует, что $\vec{x}-\vec{y}=0$ или $\vec{x}=\vec{y}$. \pause 
\item Оператор $\oper{A}$ является однозначной функцией и не может принимать различные значения. 
\end{itemize}
}
\end{proof}
}

\frame{
\frametitle{Обратный оператор}

Из доказанных свойств вытекает следующая теорема 


\begin{theorems}
\rm
\parbox{\textwidth}{
Неособенный оператор  $\oper{A}$ в $\Rn$ имеет обратный оператор такой, что:
\[
\oper{A}^{-1}(\oper{A}\vec{x}) = \vec{x},
\] \pause 
т.е. 
\[
\oper{A}\oper{A}^{-1}=\oper{A}^{-1}\oper{A}=\oper{E}.
\]
Оператор $\oper{A}^{-1}$ является линейным оператором.  \pause 
}
\end{theorems}

}

\frame{
\frametitle{Линейность обратного оператора}

\begin{proof}
\rm
\parbox{\textwidth}{
Пусть $\vec{y}=\oper{A}\vec{x}$, $\vec{u}=\oper{A}\vec{w}$,  \pause и $\oper{A}$ неособенный оператор в $\Rn$. \pause 
\[
\begin{array}{c}
\alpha\vec{y}+\beta\vec{u}= \pause \alpha\oper{A}\vec{x}+\beta\oper{A}\vec{w}= \pause \oper{A}(\alpha\vec{x}+\beta\vec{w})\\ \pause 
\Downarrow \\
\oper{A}^{-1}(\alpha\vec{y}+\beta\vec{u})=\alpha\vec{x}+\beta\vec{w}= \pause \alpha\oper{A}^{-1}\vec{y}+\beta\oper{A}^{-1}\vec{u}.
\end{array}
\]

}
\end{proof}
}

\frame{
\frametitle{Матрица обратного оператора}

\begin{theorems}
\rm
\parbox{\textwidth}{
Если в некотором базисе оператору $\oper{A}$ отвечает матрица $A$, \pause  тогда в этом же базисе оператору $\oper{A}^{-1}$ будет отвечать матрица $A^{-1}$.
}
\end{theorems}

}


\frame{
\frametitle{Матрица обратного оператора}

\begin{proof}
\rm
\parbox{\textwidth}{
Пусть $\oper{A}$ неособенный оператор в $\Rn$.  \pause Пусть $\oper{A}^{-1}$ -- обратный оператор к $\oper{A}$. \pause  Пусть $A$ -- матрица оператора $\oper{A}$, $B$ -- матрица оператора $\oper{A}^{-1}$ в некотором базисе, \pause  тогда
\[
\oper{A}^{-1}(\oper{A}\vec{x}) = \vec{x}=\oper{E}\vec{x}.
\] \pause 

Таким образом, $\oper{A}^{-1}\oper{A}=\oper{E}$, и, в матричном виде
\[
BA=E,
\]
т.е. 
\[
B=A^{-1}.
\]
}
\end{proof}
}

\frame{

\frametitle{Характеристические числа и собственные векторы}

\begin{dfn}
%\rm
\parbox{\textwidth}{
Пусть $\oper{A}$ -- линейный оператор в $\Rn$,  \pause тогда числа $\lambda$ и векторы $\vec{x}$, \pause  удовлетворяющие равенству
\[
\oper{A}\vec{x}=\lambda\vec{x} \quad (\vec{x}\neq \vec{0}),
\] \pause 
называются \alert{собственными (характеристическими) числами}  \pause и соответствующие им векторы -- \alert{собственными векторами} линейного оператора $\oper{A}$. 
}
\end{dfn}
}

\frame{
\frametitle{Харастеристический многочлен}
\parbox{\textwidth}{
\[
\oper{A}\vec{x}=\lambda\vec{x}  \pause 
\Leftrightarrow 
\oper{A}\vec{x}=(\lambda\oper{E})\vec{x} \pause 
\Leftrightarrow
\left(\oper{A}-\lambda\oper{E}\right)\vec{x}=0. \pause 
\]

Вектор $\vec{x}\neq 0$ и число $\lambda$ существует,  \pause когда $|\oper{A}-\lambda\oper{E}|$=0,  \pause или в некотором базисе 
\[
\chi(\lambda)= \pause 
|A-\lambda E|= \pause 
\begin{vmatrix}
a_{11}-\lambda & a_{12} & \ldots & a_{1n} \\
a_{21} & a_{22}-\lambda & \ldots & a_{2n} \\
\vdots & \vdots & \ddots & \vdots \\
a_{n1} & a_{n2} & \ldots & a_{nn}-\lambda 
\end{vmatrix}
=0.
\] \pause 

\begin{dfn}
%\rm
\parbox{\textwidth}{
Функция $\chi(\lambda)$, приведенного вида,  \pause называется \alert{характеристическим} \alert{многочленом} (характеристической функцией) оператора $\oper{A}$.
}
\end{dfn}
}
}

\frame{
\frametitle{Независимость характеристического многочлена от выбора базиса}
\begin{theorems}
\rm
\parbox{\textwidth}{
Вид характеристического многочлена линейного оператора $\oper{A}$ в $\Rn$ не зависит от выбора базиса.
}
\end{theorems} \pause 

\begin{proof}
\rm
\parbox{\textwidth}{
Пусть $A$ и $A'$ -- матрицы,  \pause соответствующие оператору $\oper{A}$ в различных базисах пространства $\Rn$.  \pause По свойству они подобны,  \pause значит существует $|T|\neq 0$: $A=TA'T^{-1}$,  \pause тогда
\[
|A-\lambda E|= \pause |TA'T^{-1}-T\lambda ET^{-1}|= \pause |T(A'-\lambda E)T^{-1}| = \pause  
\]
\[
=|T||A'-\lambda E||T^{-1}|=  \pause  |T||A'-\lambda E|/|T|= \pause |A'-\lambda E|.
\]
}
\end{proof}

}


\frame{
\frametitle{Алгоритм поиска собственных чисел и собственных векторов линейного оператора $\oper{A}$ в $\Rn$}
 \pause 
\begin{enumerate}
\item Выбирается базис в $\Rn$. \pause 
\item Находим матрицу $A$, соответствующую $\oper{A}$ в выбранном базисе. \pause 
\item Находим собственные значения из решения уравнения 
\[
\chi(\lambda)=0.
\] \pause 
\item Для каждого найденного $\lambda$ находим пространство решений $x$ из соотношения
\[
(A-\lambda E)x = 0.
\]
\end{enumerate}


}

\frame{
\frametitle{Свойства собственных векторов}

\begin{theorems}
\rm
\parbox{\textwidth}{
Если $\vec{x}$, $\vec{y}$ -- собственные векторы оператора $\oper{A}$,  \pause  соответствующие одному и тому же собственному числу $\lambda$,  \pause то их линейная комбинация также является собственным вектором для этого же числа $\lambda$. \pause 
}
\end{theorems}

\begin{proof}
\rm
\parbox{\textwidth}{
Из условия теоремы $\oper{A}\vec{x}=\lambda\vec{x}$,  $\oper{A}\vec{y}=\lambda\vec{y}$.  \pause Умножим первое соотношение на $\alpha$, второе на $\beta$, сложим, \pause и, пользуясь линейностью, произведем преобразования
\[
\begin{array}{c}
\alpha\oper{A}\vec{x}+\beta\oper{A}\vec{y}= \pause \alpha\lambda\vec{x}+\beta\lambda\vec{y}\\ \pause 
\oper{A}(\alpha\vec{x}+\beta\vec{y})= \pause \lambda(\alpha\vec{x}+\beta\vec{y}). \pause 
\end{array}
\]
Следовательно вектор $\alpha\vec{x}+\beta\vec{y}$ для любых $\alpha,\beta \in {\rm R}$  \pause  также является собственным вектором $\oper{A}$, соответствующим собственному числу $\lambda$.   
}
\end{proof}
}




\frame{
\frametitle{Свойства собственных векторов}

\begin{theorems}
\rm
\parbox{\textwidth}{
Собственные векторы соответствующие различным собственным числам являются линейно независимыми.
}
\end{theorems} \pause 
\begin{proof}
\rm
\parbox{\textwidth}{
Пусть $\vec{x}_1$, \ldots, $\vec{x}_k$ -- собственные векторы,  \pause соответствующие собственным числам $\lambda_1$,\ldots,$\lambda_k$,  \pause причем $\lambda_i\neq\lambda_j$ ($i\neq j$). \pause 

Предположим, что существует линейная комбинация 
\[
\vec{l}(\alpha_i,\vec{x}_j)=\alpha_1\vec{x}_1+\alpha_2\vec{x}_2 + \ldots + \alpha_k\vec{x}_k=0 \quad (\alpha_k\neq 0).
\] \pause 
Подействуем линейными операторами $\oper{A}-\lambda_j\oper{E}$ ($j=\overline{1,k-1}$) на $\vec{l}(\alpha_i,\vec{x}_j)$: \pause 
\[
(\oper{A}-\lambda_{k-1}\oper{E})\ldots(\oper{A}-\lambda_{2}\oper{E})(\oper{A}-\lambda_{1}\oper{E})\vec{l}(\alpha_i,\vec{x}_j).
\]

}
\end{proof}
}


\frame{
\frametitle{Свойства собственных векторов}
\begin{proof}
\rm
\parbox{\textwidth}{

\[
\begin{array}{l}
(\oper{A}-\lambda_1\oper{E})(\alpha_1\vec{x}_1+\alpha_2\vec{x}_2 + \ldots + \alpha_k\vec{x}_k)= \\ \pause 
=\alpha_1(\oper{A}-\lambda_1\oper{E})\vec{x}_1+ \pause \alpha_2(\oper{A}-\lambda_1\oper{E})\vec{x}_2+\ldots+ \pause 
\alpha_k(\oper{A}-\lambda_1\oper{E})\vec{x}_k=\\ \pause 
=\alpha_2(\lambda_2-\lambda_1)\vec{x}_2+ \pause \alpha_3(\lambda_3-\lambda_1)\vec{x}_3+ \pause \ldots+
\alpha_k(\lambda_k-\lambda_1)\vec{x}_k=0\\ \pause 
\\
(\oper{A}-\lambda_2\oper{E})(\alpha_2(\lambda_2-\lambda_1)\vec{x}_2+\alpha_3(\lambda_3-\lambda_1)\vec{x}_3+ \ldots+
\alpha_k(\lambda_k-\lambda_1)\vec{x}_k)=\\ \pause 
=
\alpha_3(\lambda_3-\lambda_1)(\lambda_3-\lambda_2)\vec{x}_3+ \pause \ldots+
\alpha_k(\lambda_k-\lambda_1)(\lambda_k-\lambda_2)\vec{x}_k)=0\\ \pause 
\vdots \\
\alpha_k(\lambda_k-\lambda_1)(\lambda_k-\lambda_2)\ldots(\lambda_k-\lambda_{k-1}) \vec{x}_k=0
\end{array}
\] \pause 
Последнее равенство невозможно в силу того, \pause  что $\alpha_k\neq 0$ (из предположения от противного); \pause  $\lambda_k-\lambda_i \neq 0$, т.к. $\lambda_i \neq \lambda_j$ ($i\neq j$);  \pause $\vec{x_k}\neq 0$ по определению собственного вектора.  \pause 

Следовательно векторы $\vec{x}_1$, \ldots, $\vec{x}_k$ линейно независимы. 
}
\end{proof}
}

\frame{
\frametitle{Линейный оператор простой структуры}
\rm
\begin{dfn}
\parbox{\textwidth}{
Линейный оператор $\oper{A}$ в пространстве $\Rn$ \pause  называется \alert{оператором простой структуры},  \pause если он имеет $n$ различных собственных чисел.
}
\end{dfn} \pause 

\begin{exampleblock}{Вид оператора простой структуры} \pause 
\parbox{\textwidth}{
Пусть линейный оператор $\oper{A}$ в пространстве $\Rn$  \pause имеет $n$ различных собственных векторов $\vec{x}_i$,  \pause соответствующие различным собственным числам $\lambda_i$ ($i=\overline{1,n}$).  \pause По только что доказанной теореме система из $n$ векторов $\vec{x}_i$ ($i=\overline{1,n}$) линейно независима,  \pause а значит является базисом.  \pause Получим вид матрицы $A$ в этом базисе.
}
\end{exampleblock}
}

\frame{
\frametitle{Вид оператора простой структуры}

\begin{exampleblock}{\parbox{\textwidth}{Разложение образов базисных векторов по базису пространства $\Rn$}}
\pause
\[
\begin{array}{ccl}
\oper{A}\vec{x}_1 &  =  & \lambda_1 \cdot \vec{x}_1 + 0\cdot\vec{x}_2+\ldots+0\cdot\vec{x}_n,\\ 
\pause
\oper{A}\vec{x}_2 &  =  & 0 \cdot \vec{x}_1 + \lambda_2\cdot\vec{x}_2+\ldots+0\cdot\vec{x}_n,\\ 
\vdots &  &  \\
\pause
\oper{A}\vec{x}_n &  =  & 0 \cdot \vec{x}_1 + 0\cdot\vec{x}_2+\ldots+\lambda_n\cdot\vec{x}_n.
\end{array}
\]
\end{exampleblock}\pause

\begin{exampleblock}{Матрица оператора $\oper{A}$ в базисе $\vec{x}_i$ ($i=\overline{1,n}$)}
\pause
\[
A=\left(
\begin{array}{cccc}
\lambda_1 & 0 & \ldots & 0 \\
0 & \lambda_2 & \ldots & 0 \\
\vdots & \vdots & \ddots & \vdots\\
0 & 0 & \ldots & \lambda_n \\
\end{array}\right).
\]
\end{exampleblock}


}


\frame{
\frametitle{Свойства собственных векторов}

\begin{theorems}[без доказательства]
\rm
\parbox{\textwidth}{
Все характеристические числа вещественной симметричной матрицы вещественны, \pause  и существует базис из ортонормированных собственных векторов. \pause 
}
\end{theorems}

\begin{exampleblock}{Пример.}
\[
A=\left(
\begin{array}{cc}
1 & 2 \\
2 & 1
\end{array}
\right). \pause 
\]
\[
\chi(\lambda)=
\left|
\begin{array}{cc}
1-\lambda & 2 \\
2 & 1-\lambda
\end{array}
\right|= \pause 
(1-\lambda)^2-4=0  \pause \Leftrightarrow \lambda_{1}=3,\quad\lambda_{2}=-1.
\] \pause 
\[
\begin{array}{ccll}
\vec{g}_1 & = &  (1/\sqrt{2},& 1/\sqrt{2}), \\ \pause 
\vec{g}_2 & = & (-1/\sqrt{2},& 1/\sqrt{2}).  \pause 
\end{array}
\]
\[
\vec{g}_1\cdot\vec{g}_2 = 0,\quad
\vec{g}_1\cdot\vec{g}_1 = 1,\quad
\vec{g}_2\cdot\vec{g}_2 = 1.
\]
\end{exampleblock}


}


\end{document}
%\begin{columns}и
%\begin{column}{0.5\textwidth}
%\end{column}
%
%\begin{column}{0.5\textwidth}
%\end{column}
%
%
%\end{columns}
