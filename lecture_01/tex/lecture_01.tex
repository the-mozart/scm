\documentclass{beamer}

\usepackage{beamerthemesplit}
\usetheme{Singapore} %Copenhagen}
%\usecolortheme{whale}

\usepackage[T2A]{fontenc}
\usepackage[utf8]{inputenc}
\usepackage[russian]{babel}


\usepackage{textcomp}
\usepackage{amssymb,amsmath}
%\usepackage{animate}
%\usepackage{longtable}
\usepackage{xcolor}

\newcounter{N}

%% Форматирование окружения itemize
%\usepackage{ragged2e}
%\let\olditem\item
%\renewcommand\item{\olditem\justifying}


\title[]{Сведения из теории матриц. Линейные преобразования векторных пространств и их свойства.
}

\author[]{ {\em Верещагин Антон Сергеевич}
	\\
	канд. физ.-мат. наук, доцент\\
	\bigskip
	Кафедра аэрогидродинамики ФЛА НГТУ
}

%
%\date[]{Новосибирск -- 2016}

\newtheorem{dfn}{Определение}  
\newtheorem{theorems}{Теорема}  

\newcommand{\Rn}{\mathrm{R}^n}
\newcommand{\Sm}{\mathrm{S}^m}
\newcommand{\Ql}{\mathrm{Q}^l}

\newcommand{\Rd}[1]{\mathbb{R}^{#1}}
\newcommand{\Vn}{\mathrm{V}^n}


\newcommand{\oper}[1]{\mathcal{#1}}

\begin{document}
\frame{\titlepage}


\frame{
\frametitle{Аннотация}
\parbox{\textwidth}{
Векторное пространство. Отображение n-мерного вектора в m-мерный. Линейные операторы. Матрица, соответствующая линейному оператору. Сложение и умножение линейных операторов. Преобразования координат. Эквивалентные матрицы.
}
}


\frame{
\frametitle{ Определение группы }
\begin{dfn}
\rm
\parbox{\textwidth}{
{\bf Группой} называется упорядоченная двойка $\{X,\cdot\}$, где $X$ -- множество элементов, а $\cdot:X^2 \to X$ -- операция между элементами множества $X$, обладающая следующими свойствами:
\begin{enumerate}
\item  $(a\cdot b) \cdot c = a\cdot (b \cdot c)$ ($\forall a,b,c \in X$)
\item $\exists 0\in X$: $0+x=x+0=x$ ($\forall x\in X$)
\item $\forall x\in X$ $\exists (-x)\in X$: $x\cdot(-x)=(-x)\cdot x = 0$
\end{enumerate}

\medskip
Если выполнено дополнительно следующее свойство, то группа называется {\bf абелевой}:
\begin{enumerate}
\item[4.] $x\cdot y=y\cdot x$ ($\forall x,y \in X$)
\end{enumerate}
}
\end{dfn}
}

\frame{
\frametitle{Векторное пространство}

\begin{dfn}
\rm
\parbox{\textwidth}{
{\bf  Векторным пространством} называется упорядоченная тройка $\{V,R,\cdot\}$, где $V$ -- абелева группа  по сложению с элементами, которые будем обозначать $\vec{x}$ и называть векторами. $R$ -- поле скаляров. $\cdot:R\times V \to V$ -- однозначно определенная операция умножения скаляра на вектор. При этом должны выполнятся следующие условия:
}

\begin{enumerate}
% \pause
%  \item $\vec{x}+\vec{y}=\vec{y}+\vec{x}$
% \pause
% \item $(\vec{x}+\vec{y})+\vec{z}=\vec{x}+(\vec{y}+\vec{z})$
% \pause
% \item $\exists\vec{0}\in V$ такой, что $\vec{0}+\vec{x}=\vec{x}$
 \pause
 \item $1\cdot\vec{x}=\vec{x}$\quad $\forall x\in V$
 \pause
 \item $\alpha\cdot(\beta\cdot\vec{x})=(\alpha\beta)\cdot\vec{x}$
 \pause
 \item ($\alpha+\beta)\cdot\vec{x}=\alpha\cdot\vec{x}+\beta\cdot\vec{x}$
 \pause
 \item $\alpha\cdot(\vec{x}+\vec{y})=\alpha\cdot\vec{x}+\alpha\cdot\vec{y}$
\end{enumerate}

\end{dfn}


}

%\frame{
%\frametitle{Примеры векторных пространств}
%
%
%}


\frame{
\frametitle{Линейная зависимость и линейная независимость векторов}
\only<1>{
\begin{dfn}
\rm 
Линейной комбинацией векторов $\vec{x}_1$,\ldots,$\vec{x}_n$ c коэффициетами $\alpha_1$, \ldots, $\alpha_n$ называется следующая сумма
\begin{equation*}
\alpha_{1}\vec{x}_1+\alpha_2\vec{x}_2+\ldots+\alpha_n\vec{x}_n
\end{equation*}
\end{dfn}
}


\begin{dfn}<2->
\rm 
Векторы $\vec{x}$, $\vec{y}$, $\vec{z}$, \ldots называются {\bf линейно зависимыми}, ecли существуют такие скаляры $\alpha$, $\beta$, $\gamma$, \ldots, причем один из них отличен от $0$, такие что
\[
\alpha \vec{x} + \beta \vec{y} + \gamma \vec{z} + \ldots = \vec{0}.
\]
\end{dfn}

\begin{dfn}<3->
\rm 
Векторы $\vec{x}$, $\vec{y}$, $\vec{z}$, \ldots называются {\bf линейно независимыми}, если из равенства
\[
\alpha \vec{x} + \beta \vec{y} + \gamma \vec{z} + \ldots = \vec{0}
\]
следует, что все $\alpha=\beta=\gamma=\ldots=0$.
\end{dfn}

}

%\frame{
%\frametitle{Пример линейно зависимых векторов}
%
%}
%
%\frame{
%\frametitle{Пример линейно независимых векторов}
%
%}


\frame{
\frametitle{Размерность и базис векторного пространства}
\begin{dfn}<1->
\rm
\parbox{\textwidth}{
Векторное пространство $\mathrm{V}$ называется {\bf $\mathbf{n}$-мерным}, если в нем существуют $n$ линейно независимых векторов, а любые \mbox{$n+1$} будут линейно зависимы. $n$-мерное векторное пространство обозначается $\Vn$.
}
\end{dfn}

\bigskip
\begin{dfn}<2->
\rm
\parbox{\textwidth}{
{\bf Базисом} $n$-мерного пространства $\Vn$ называется система любых $n$ линейно независимых векторов $\vec{e}_1$, $\vec{e}_2$, \ldots,$\vec{e}_n$.
}
\end{dfn}

}

\frame{
\frametitle{Пример конечномерного пространства}

\begin{exampleblock}{Пространство $\Rd{3}$}
Рассмотрим четыре вектора в пространстве $\Rd{3}$
\begin{equation*}
\begin{array}{lcl}
\vec{x_i} & = & \{x_i^1,x_i^2,x_i^3\}\\
\end{array} (i=1,\ldots,4)
\end{equation*}


\parbox{\textwidth}{
Ранг матрицы, составленный из координат этих векторов не может превышать $3$, значит любые $4$ вектора в $\Rd{3}$ будут всегда линейно зависимы.
}
\end{exampleblock}

\begin{exampleblock}{Базис в $\Rd{3}$}
Базисом в $\Rd{3}$ будут три любые линейно независимых вектора, например,
\begin{equation*}
\begin{array}{lcl}
\vec{e_1} & = & \{1,0,0\},\\
\vec{e_2} & = & \{0,1,0\},\\
\vec{e_3} & = & \{0,0,1\}.\\
\end{array}
\end{equation*}


\end{exampleblock}

}

\frame{
\frametitle{Разложение вектора по базису пространства}


\begin{theorems}
\rm
\parbox{\textwidth}{
В вектороном $n$-мерном пространстве $\Vn$ каждый вектор может быть единственным способом представлен в виде линейной комбинации векторов базиса.
}
\end{theorems}

}

\frame{
\frametitle{Разложение вектора по базису пространства}

\begin{proof}
\parbox{\textwidth}{
\begin{itemize}

\item Рассмотрим базис векторного пространства $\Vn$ $\vec{e}_1$, \ldots, $\vec{e}_n$ и произвольный вектор $\vec{x}\in \Vn$.

\item<2-4> Система векторов $\vec{x}$, $\vec{e}_1$, \ldots, $\vec{e}_n$ состоит из $n+1$ вектора, поэтому является линейно зависимой. 

\item<3-4>
Следовательно существуют коэффициенты $\alpha_1$,\ldots,$\alpha_n$, $\beta$, причем один из $\alpha_i \neq 0$ (пусть $i=1$) и $\beta\neq 0$, такие что
\begin{equation*}
\alpha_1\vec{e}_1+\alpha_2\vec{e}_2+\ldots+\alpha_n\vec{e}_n + \beta \vec{x} = 0.
\end{equation*}

\item<4>
Следовательно 
\begin{equation*}
\vec{x}=-\frac{\alpha_1}{\beta}\vec{e_1}-\frac{\alpha_2}{\beta}\vec{e_2}-\ldots-\frac{\alpha_n}{\beta}\vec{e_n}.
\end{equation*}




\end{itemize}
}

\end{proof}

}


\frame{
\frametitle{Единственность разложения по базису}

\begin{proof}
\parbox{\textwidth}{
\begin{itemize}

\item<1->
Пусть вектор 
\[
\vec{x}=\alpha_1\vec{e}_1+\alpha_2\vec{e}_2+\ldots+\alpha_n\vec{e}_n=\beta_1\vec{e}_1+\beta_2\vec{e}_2+\ldots+\beta_n\vec{e}_n,
\]
где $\alpha_i\neq\beta_i$ для некоторого $i$.

\item<2-> %Выражение
$
(\alpha_1-\beta_1)\vec{e}_1+(\alpha_2-\beta_2)\vec{e}_2+\ldots+(\alpha_n-\beta_n)\vec{e}_n=0,
$
где $\alpha_i-\beta_i\neq 0$.

\item<3->
Это невозможно в силу линейной независимости базиса $\vec{e}_i$ $(i=1,\ldots,n)$.
\end{itemize}
}
\end{proof}
\onslide<4>{
\begin{dfn}
\parbox{\textwidth}{
\rm
Коэффициенты в разложении вектора $\vec{x}$ по базису $\vec{e}_i$ называются координатами вектора $\vec{x}$ в базисе $\vec{e}_i$ $(i=1,\ldots,n)$.
}
\end{dfn}
}

}

\frame{
\frametitle{Теорема о линейной независимости векторов}

\begin{theorems}[о линейной независимости векторов]
\rm
\parbox{\textwidth}{
Для того, чтобы векторы $\vec{x}_1$, $\vec{x}_2$, \ldots, $\vec{x}_m$ в пространстве $R^n$ были линейно независимы, необходимо и достаточно, чтобы ранг матрицы, составленной из координат этих векторов в произвольном базисе, был равен числу этих векторов. В противном случае они линейно зависимы.
}
\end{theorems}
}


\frame{
\frametitle{Линейный оператор}
\begin{dfn}
\rm
Отображение одного конечномерного пространства в другое  $\oper{A}:\Rn\rightarrow \Sm$ называется линейным, если 
\[
 \oper{A}(\vec{x}+\vec{y})=\oper{A}\vec{x}+\oper{A}\vec{y},\quad \oper{A}(\alpha\vec{x})=\alpha\oper{A}\vec{x}.
\]
\end{dfn}
 
}


\frame{
\frametitle{Матрица линейного оператора}
\begin{theorems}
\rm
\parbox{\textwidth}{
\onslide<1->{
Пусть $\oper{A}$ -- линейное преобразования $n$-мерного векторного пространства $\Rn$ в $m$-мерное векторное пространство $\Sm$ 
\[
\oper{A}:\Rn\rightarrow \Sm.
\]
}
\onslide<2->{
Пусть $\vec{e}_i$ $(i=1,\ldots,n)$ -- базис $\Rn$, а $\vec{g}_k$ $(k=1,\ldots,m)$ -- базис $\Sm$.\\
}
\onslide<3->{
Пусть для произвольного вектора $\vec{x}\in\Rn$ $\vec{y}=\oper{A}\vec{x} \in \Sm$. \\
}
\onslide<4->{
Тогда существует матрица $A$ размера $m\times n$ такая, что
\[
y=Ax,
\]
где $y$ -- вектор столбец, составленный из координат вектора $\vec{y}$ в базисе $\vec{g}_k$, $x$ -- вектор столбец, составленный из координат вектора $\vec{x}$ в базисе $\vec{e}_i$.
}
}
\end{theorems}
 
}

\frame{
\frametitle{Доказательство теоремы представлении линейного оператора}

\begin{proof}
\begin{itemize}
\item[] Пусть $\vec{y}=\oper{A}\vec{x}\in\Sm$ для некоторого $\vec{x}\in \Rn$. 
\pause
\item[] Тогда $\vec{y}=\sum\limits_{k=1}^m y_k\vec{g}_k$, $\vec{x}=\sum\limits_{i=1}^n x_i\vec{e}_i$.
\pause
\item[] В силу линейности оператора $\oper{A}$:
\[
\vec{y}=\oper{A}\vec{x}=\oper{A}\left(\sum\limits_{i=1}^n x_i\vec{e}_i\right)=\sum\limits_{i=1}^n x_i (\oper{A}\vec{e}_i).
\]
\pause
\item[] Т.к. $\oper{A}\vec{e}_i\in \Sm$, тогда существуют такие числа $\alpha_{ij}$%($1\leq i \leq n$, $1\leq j\leq m$)
, что
\[
\oper{A}\vec{e}_i=\sum\limits_{j=1}^m \alpha_{ij}\vec{g}_j\quad
(i=1,\ldots,n).
\]
\end{itemize}


\end{proof}

}

\frame{
\frametitle{Доказательство теоремы представлении линейного оператора}

\begin{proof}
\begin{itemize}
\item[] Тогда
\[
\vec{y}=\sum\limits_{i=1}^n x_i (\oper{A}\vec{e}_i)=\sum\limits_{i=1}^n x_i \left( \sum\limits_{j=1}^m \alpha_{ij}\vec{g}_j\right)=
\sum\limits_{j=1}^m \left(\sum\limits_{i=1}^n x_i \alpha_{ij}\right) \vec{g}_j.
\]
\pause
\item[] Из единственности представления вектора $\vec{y}$ по базису $\vec{g}_k$ ($k=1,\ldots,m$) следует, что
\[
y_j=\sum\limits_{i=1}^n x_i \alpha_{ij}.
\]


\end{itemize}


\end{proof}

}


\frame{
\frametitle{Иллюстрация теоремы о представлении линейного оператора}

\begin{itemize}
\item[] Пусть $x$ -- вектор столбец, составленный из координат, вектора $\vec{x}$ в базисе $\vec{e}_i$ $x_i$.
\pause
\item[] Пусть $y$ -- вектор столбец, составленный из координат, вектора $\vec{y}$ в базисе $\vec{g}_k$ $y_k$.
\pause
\item[] Пусть $A=(a_{ij})$ -- $m\times n$ матрица, составленная из координат образов векторов $\oper{A}\vec{e}_i$ по столбцам ($a_{ij}=\alpha_{ji}$)
\[
A = \left(
\begin{array}{c|c|c|c}
\oper{A}\vec{e}_1 & \oper{A}\vec{e}_2 & \ldots & \oper{A}\vec{e}_n\\
\end{array}\right)=
\left(
\begin{array}{cccc}
\alpha_{11} & \alpha_{21} & \ldots & \alpha_{n1} \\
\alpha_{12} & \alpha_{22} & \ldots & \alpha_{n2} \\
\vdots & \vdots & \ddots & \vdots\\
\alpha_{1m} & \alpha_{2m} & \ldots & \alpha_{nm} \\
\end{array}\right)
\]
\end{itemize}

}

\frame{
\frametitle{Иллюстрация теоремы о представлении линейного оператора}

\begin{itemize}
\item[] Тогда 
\[
\left( 
\begin{array}{c}
y_1 \\
y_2\\
\vdots\\
y_m
\end{array}\right)=
\left(
\begin{array}{cccc}
a_{11} & a_{12} & \ldots & a_{1n} \\
a_{12} & a_{22} & \ldots & a_{2n} \\
\vdots & \vdots & \ddots & \vdots\\
a_{m1} & a_{m2} & \ldots & a_{mn} \\
\end{array}\right)
\left( 
\begin{array}{c}
x_1 \\
x_2\\
\vdots\\
x_n
\end{array}\right).
\]
\end{itemize}

}

\frame{
\frametitle{Сложение операторов}

\begin{dfn}
\rm 
\begin{itemize}
\item[]
Пусть $\oper{A}$, $\oper{B}$ -- линейные операторы, действующие из пространства $\Rn$ в $\Sm$.
\pause 
\item[]
Оператор $\oper{C}:\Rn\rightarrow\Sm$ называется суммой $\oper{A}$ и $\oper{B}$ и обозначается
\[
\oper{C}=\oper{A}+\oper{B}, 
\]
\pause
\item[]
тогда и только тогда, когда 
\[
\forall \vec{x}\in R^n\quad \oper{C}\vec{x}=\oper{A}\vec{x} +\oper{B}\vec{x}.
\]
\end{itemize}
\end{dfn}


}


\frame{
\frametitle{Произведение операторов}

\begin{dfn}
\rm 
\begin{itemize}
\item[]
Пусть $\oper{A}:\Rn\rightarrow\Sm$, $\oper{B}:\Ql\rightarrow\Rn$ -- линейные операторы.
\pause 
\item[]
Оператор $\oper{C}:\Ql\rightarrow\Sm$ называется произведением $\oper{A}$ и $\oper{B}$ и обозначается
\[
\oper{C}=\oper{A}\oper{B}, 
\]
\pause
\item[]
тогда и только тогда, когда 
\[
\forall \vec{x}\in \Ql \quad \oper{C}\vec{x}=\oper{A}(\oper{B}\vec{x}).
\]
\end{itemize}
\end{dfn}


}


\frame{
\frametitle{Связь между координатами векторов в различных базисах}
\begin{itemize}
\item[]
Пусть $\vec{e}_i$, $\vec{g}_j$ ($i,j=1,\ldots,n$) два различных базиса векторного пространства $\Rn$.
\pause
\item[] 
Существуют числа $t_{ij}$  и матрица $T=(t_{ij})$ ($i,j=1,\ldots,n$) ($|T|\neq 0$), такая что
\pause
\item[]
\[
\vec{e}_i=\sum\limits_{j=1}^n t_{ij} \vec{g}_j\quad
(i=1,\ldots,n).
\]
\pause
\item[]
Пусть $\vec{x} = \sum\limits_{i=1}^n x_i \vec{e}_i=\sum\limits_{i=1}^n x'_i \vec{g}_i$ -- два различных представления одного вектора в различных базисах.
\end{itemize}

}

\frame{
\frametitle{Связь между координатами векторов в различных базисах}
\begin{itemize}
\item[] Рассмотрим 
\[
\vec{x}=\sum\limits_{i=1}^n x_i \vec{e}_i = \sum\limits_{i=1}^n x_i \left( \sum\limits_{j=1}^n t_{ij} \vec{g}_j \right) = \sum\limits_{j=1}^n \left( \sum\limits_{i=1}^n x_i t_{ij} \right)\vec{g}_j.
\]
\pause
\item[]
Из единственности разложения $\vec{x}$ по базису $\vec{g}_j$ следует, что
\[
x'_j= \sum\limits_{i=1}^n x_i t_{ij},
\]
\pause
\item[]
что в матричном виде запишется
\[
x'=T^{\rm t}x,
\]
где $x'$, $x$ -- вектор столбцы координат вектора $\vec{x}$ в соответствующих базисах.

\end{itemize}

}

\frame{
\frametitle{Эквивалентные матрицы}

\begin{dfn}
\rm
Матрицы $A$, $B$ размера $m\times n$ называются {\bf эквивалентными}, если существуют матрица $P$ размера $m\times m$ ($|P|\neq 0$) и матрица $Q$ ($|Q|\neq 0$) размера $n\times n$ такие, что
\[
A = P B Q.
\]
\end{dfn}

}

\frame{
\frametitle{Связь между матричным представлением линейного оператора в различных базисах}

\begin{theorems}
\rm
Матрицы, соответвующие линейному оператору $\oper{A}:\Rn\rightarrow\Sm$, в различных базисах пространств $\Rn$ и $\Sm$ эквивалентны.
\end{theorems}

}

\frame{
\begin{proof}

\begin{itemize}

\item[]
Пусть $\vec{e}_1,\ldots,\vec{e}_n$ и $\vec{e}'_1,\ldots,\vec{e}'_n$ базис пространства $\Rn$, а векторы $\vec{g}_1,\ldots,\vec{g}_m$ и $\vec{g}'_1,\ldots,\vec{g}'_m$ базисы пространства $\Sm$.
\pause
\item[]
Пусть $\oper{A}$ -- линейной преобразование $\Rn$ в $\Sm$. 

\pause
\item[]
Пусть $A$ -- матрица преобразования, соответствующая оператору  $\oper{A}$  в базисах $e$ и $g$, а $A'$ -- в базисах $e'$ и $g'$.

\pause
\item[]
Пусть для некоторых $\vec{x}$ из $\Rn$ и $\vec{y}$ из $\Sm$ 
\[
	\vec{y}=\oper{A}\vec{x}.
\]

\pause
\item[]
Тогда в матричной записи в соответствующих базисах $y=Ax$ и $y'=A'x'$

\pause
\item[]
Пусть $Q$ -- $n\times n$-матрица перехода между координатами векторов в установленных базисах $x=Qx'$ в $\Rn$, 

\pause
\item[] а $N$ -- $m\times m$-матрица перехода между координатами вектора в $\Sm$ $y=Ny'$.


\end{itemize}

\end{proof}
}

\frame{
\begin{proof}

\begin{itemize}

\item[] 
С одной стороны
\[
y=Ax=AQx'
\]

\pause
\item[]
С другой стороны
\[
y=Ny'=NA'x'.
\]

\pause
\item[] Таким образом,
\[
\forall x'\quad
NA'x'=AQx'.
\]

\pause
\item[]
Следовательно 
\[
A'=N^{-1}AQ.
\]
\end{itemize}

\end{proof}
}

\frame{
\frametitle{Теорема об эквивалентности матриц}
\begin{theorems}[об эквивалентности матриц]
\rm
\parbox{\textwidth}{
Для того чтобы две прямоугольные матрицы $A$ и $B$ одинаковых размеров $m\times n$ были эквивалентны, необходимо и достаточно, чтобы их ранги совпадали $r_A=r_B$.
}
\end{theorems}
}

\end{document}
%\begin{columns}
%\begin{column}{0.5\textwidth}
%\end{column}
%
%\begin{column}{0.5\textwidth}
%\end{column}
%
%
%\end{columns}
