\documentclass{beamer}

\usepackage{beamerthemesplit}
\usetheme{Singapore} %Copenhagen}
%\usecolortheme{whale}

\input{../../include/preamble.inc} 
\input{../../include/definitions.inc} 
\input{../../include/author.inc} 


\title[]{Основные свойства аффинных ортогональных тензоров}

\begin{document}

\frame{\titlepage}


\frame{
\frametitle{Аннотация}
\parbox{\textwidth}{
Классификация тензоров. Теорема о полном тензоре. Теорема о существовании обратного тензора. Главные значения тензора. Инварианты тензора. Бискалярное произведение. Тензорное поле. Дивергенция тензора.
}
}

\frame{
	\frametitle{ Классификация тензоров }
	
	\begin{exampleblock}{}
		\parbox{\textwidth}{
			Пусть для  выбранного тензора $\tensor{A}$ и произвольного вектора $\vec{r}$
			\[
			\vec{r'} = \tensor{A}\cdot\vec{r},
			\]
			тогда возможны следующие варианты:
			\begin{itemize} 
				\item<1-> все $\vec{r'}$ равны $0$, тогда $\tensor{A}$ -- \alert{нулевой};
				\item<2-> все $\vec{r'}$ лежат на одной прямой, тогда $\tensor{A}$ -- \alert{линейный};
				\item<3-> все $\vec{r'}$ лежат в одной плоскости, тогда $\tensor{A}$ -- \alert{планарный};
				\item<4-> $\vec{r'}$ описывают все векторы, тогда $\tensor{A}$ -- \alert{полный};
			\end{itemize}
			
		}
	\end{exampleblock}

	\only<5>{
	\begin{exampleblock}{Задача}
		\parbox{\textwidth}{
			Привести пример тензора каждого типа и обосновать. 
		}
	\end{exampleblock}
	}
	
}

\frame{
	\frametitle{Теорема о полном тензоре}
	
	\begin{theorems}[о полноте тензора]
	\parbox{\textwidth}{
		Для того, чтобы тензор $\tensor{\Pi}$ был полным необходимо и достаточно, чтобы его определитель был отличен от $0$.	
	}	
	\end{theorems} \pause 

	\begin{proof}
	\only<2>{		
	\parbox{\textwidth}{
	($\Rightarrow$) Пусть тензор $\tensor{\Pi}$ полный, тогда для любого вектора $\vec{r'} \in \Rt$ существует вектор $\vec{r} \in \Rt$, такой что $\vec{r'} = \tensor{\Pi}\cdot\vec{r}$. Для фиксированной системы координат это эквивалентно матричному равенству
	\[
	\begin{Bmatrix}
	p_{11} & p_{12} & p_{13} \\
	p_{21} & p_{22} & p_{23} \\
	p_{31} & p_{32} & p_{33} \\
	\end{Bmatrix} 
	\begin{pmatrix}
	r_{1} \\
	r_{2} \\
	r_{3} \\
	\end{pmatrix}=
	\begin{pmatrix}
	r'_{1} \\
	r'_{2} \\
	r'_{3} \\
	\end{pmatrix},
	\]
	где $r'_i$, $r_j$ и $p_{ks}$ -- координаты соответствующих векторов и компоненты тензора в выбранной системе координат. Полученная система линейных уравнений имеет решение при любой правой части, следовательно $D(\tensor{\Pi}) \neq 0$.
	}
	}	

	\only<3>{		
	\parbox{\textwidth}{
		($\Leftarrow$) Пусть $D(\tensor{\Pi}) \neq 0$, тогда система линейных уравнений
		\[
		\begin{Bmatrix}
		p_{11} & p_{12} & p_{13} \\
		p_{21} & p_{22} & p_{23} \\
		p_{31} & p_{32} & p_{33} \\
		\end{Bmatrix} 
		\begin{pmatrix}
		r_{1} \\
		r_{2} \\
		r_{3} \\
		\end{pmatrix}=
		\begin{pmatrix}
		r'_{1} \\
		r'_{2} \\
		r'_{3} \\
		\end{pmatrix},
		\]
		имеет решение $r_j$ ($j=\overline{1,3}$) при любых значениях правой части $r'_i$ ($i=\overline{1,3}$)
		где $r'_i$, $r_j$ и $p_{ks}$ -- координаты векторов и компоненты тензора в фиксированной системе координат. 
		Таким образом, у каждого вектора $\vec{r'}$ есть прообраз $\vec{r}$ такой, что
		$
		\vec{r'} = \tensor{\Pi} \cdot \vec{r}.
		$
		Следовательно тензор $\tensor{\Pi}$ -- полный.
	}
	}
	\end{proof}
	
	
}

\frame{
	\frametitle{Обратный тензор}
	
	\begin{dfn}
		Если для тензора $\tensor{\Pi}$ существует тензор $\tensor{B}$ такой, что
		\[
		\tensor{B} \cdot \tensor{\Pi} = \tensor{\Pi} \cdot \tensor{B} = \tensor{I},
		\]
		тогда тензор $\tensor{B}$ называется обратным тензором и обозначается $\tensor{\Pi}^{-1}$.
	\end{dfn}
	
}

\frame{
	\frametitle{Теорема о существовании обратного тензора}
	\begin{theorems}
		\parbox{\textwidth}{
		Полнота тензора есть необходимое и достаточное условие существования обратного тензора.
		}

	\end{theorems}

	\begin{proof}
		\parbox{\textwidth}{
		Доказательство очевидно и вытекает из правила произведения тензоров как матриц и теоремы о полноте тензора.			
		}
	\end{proof}
}

\frame{
	\frametitle{Главные значения тензора}
	
	\begin{dfn}
		\parbox{\textwidth}{
	
		Если для заданного тензора $\tensor{\Pi}$, вектора $\vec{r}$ и числа $\lambda$ справедливо равенство
		\[
		\tensor{\Pi} \cdot \vec{r} = \lambda \vec{r} \quad (\vec{r} \neq 0),
		\]
		то говорят, что $\lambda$ -- главное значение тензора $\tensor{\Pi}$, а $\vec{r}$ -- собственный вектор.
		}		
	\end{dfn}

	\begin{exampleblock}{Пояснения}
		\parbox{\textwidth}{
			Так как тензоры и векторы перемножаются по таким же законам как и матрицы, то главные значения тензора и его собственные векторы аналогичны собственным значениям и векторам соответствующим матрице тензора $\tensor{\Pi}$ в выбранной системе координат и не зависят от неё.
		}
	\end{exampleblock}
}

\frame{
	\frametitle{Инварианты тензора}
	
	\begin{dfn}
		\parbox{\textwidth}{
		Характеристическим многочленом тензора $\tensor{\Pi}$ называется функция
		\[
		\chi(\lambda) = D(\tensor{\Pi} - \lambda I) = -\lambda^3 + I_1 \lambda^2 - I_2 \lambda + I_3.
		\]
		
		Величины $I_1$, $I_2$, $I_3$ -- называются \alert{инвариантами} тензора $\tensor{\Pi}$.
		}
	\end{dfn}\pause

	\begin{exampleblock}{Независимость от системы координат}
		\parbox{\textwidth}{
			Величины $I_1$, $I_2$, $I_3$ не зависят от выбора системы координат, т.к.
			\[
			\chi(\lambda) = D(\tensor{\Pi} - \lambda I) =  \det(\Pi' - \lambda I) = 
			\det (Q^T \Pi Q - \lambda E) =
%			\det (Q^T \Pi Q - \lambda Q^T Q) =
			\]
			\[
			=
			\det (Q^T (\Pi - \lambda  E) Q)=
			\det Q^T \det(\Pi - \lambda  E) \det Q = \det(\Pi - \lambda  E),
			\]
			где $\Pi'$ и $\Pi$ компоненты тензора $\tensor{\Pi}$  в различных ортогональных системах координат, $Q$ -- матрица перехода ($Q^T = Q^{-1}$).
		}
	\end{exampleblock}

}

\frame{
	\frametitle{Формулы для вычисления инвариантов}
		\begin{exampleblock}{Свойство}
		\parbox{\textwidth}{
			Если $\Pi=(p_{ij})_{1\leq i,j \leq 3}$ матрица компонент тензора $\tensor{\Pi}$ в некотором базисе, тогда
			\[
			\begin{array}{ll}
			I_1 = p_{11} + p_{22} + p_{33} = \lambda_1 +\lambda_2 + \lambda_3,\\
			\\
			I_2 = 
			\begin{vmatrix}
			p_{11} & p_{12}\\
			p_{21} & p_{22}
			\end{vmatrix}+
			\begin{vmatrix}
			p_{22} & p_{23}\\
			p_{32} & p_{33}
			\end{vmatrix}+
			\begin{vmatrix}
			p_{11} & p_{13}\\
			p_{31} & p_{33}
			\end{vmatrix} = 
			\lambda_1\lambda_2 +\lambda_2 \lambda_3 + \lambda_1\lambda_3,\\
			\\
			I_3 =
			\begin{vmatrix}
			p_{11} & p_{12} & p_{13}\\
			p_{21} & p_{22} & p_{23}\\
			p_{31} & p_{32} & p_{33}
			\end{vmatrix}=
			\lambda_1\lambda_2\lambda_3,
			\end{array}
			\]
			где $\lambda_i$ -- собственные числа тензора $\tensor{\Pi}$.
			
		}
	\end{exampleblock}
	
}


\frame{
	\frametitle{Бискалярное произведение тензоров}
	
	\begin{dfn}
		\parbox{\textwidth}{
			\alert{Бискалярным произведением} тензоров называется первый инвариант (след) их скалярного произведения.
		}
	\end{dfn}\pause

	\begin{exampleblock}{Формула для бискалярного произведения}
		\parbox{\textwidth}{
				Пусть $\tensor{A} = (a_{ij})_{1\leq i,j \leq 3}$, $\tensor{B} = (b_{ij})_{1\leq i,j \leq 3}$, $\tensor{C} = (c_{ij})_{1\leq i,j \leq 3}$ -- три тензора, причём $\tensor{C}=\tensor{A}\cdot\tensor{B}$, тогда бискалярное произведение $\tensor{A}$ и $\tensor{B}$ равно
				\[
				A \cdot\cdot B = I_{1(C)} = \sum\limits_{i=1}^3 c_{ii} = \sum\limits_{i=1}^3 \sum\limits_{k=1}^3 a_{ik}b_{ki}.
				\]
				
				Таким образом, бискалярное произведение рассчитывается как свёртка компонентов первого и сопряжённых компонентов второго тензора.
		}
	\end{exampleblock}
	
}

\frame{
	\frametitle{Производная тензора одного аргумента}
	
	\begin{dfn}
		\parbox{\textwidth}{
			Пусть компоненты тензора $\tensor{\Pi}$ зависят от переменной $t$, т.е.
			\[
			\Pi(t) = \vec{i}_1 \vec{p}_1(t) + \vec{i}_2 \vec{p}_2(t)+  \vec{i}_3 \vec{p}_3(t),
			\]
			тогда производной тензора $\tensor{\Pi}$ по переменной $t$ называется
			\[
			\dt{\tensor{\Pi}} = \lim\limits_{\Delta t \to 0} \frac{\tensor{\Pi}(t+\Delta t) - \tensor{\Pi}(t) }{\Delta t}
			\]
		}
		
	\end{dfn}\pause
	
	\begin{exampleblock}{Производная через компоненты}
		\parbox{\textwidth}{
			
		\[
		\dt{\tensor{\Pi}} = \vec{i}_1 \dt{\vec{p}_1} + \vec{i}_2 \dt{\vec{p}_2}+  \vec{i}_3 \dt{\vec{p}_3}=
		\begin{Bmatrix}
		\dt{p_{11}} & \dt{p_{12}} & \dt{p_{13}} \\
		\dt{p_{21}} & \dt{p_{22}} & \dt{p_{23}} \\
		\dt{p_{31}} & \dt{p_{32}} & \dt{p_{33}} \\
		\end{Bmatrix}
		\]	
		}
	\end{exampleblock}
}

\frame{
	\frametitle{Свойства производной тензора}
	
	\parbox{\textwidth}{
		Для произвольного вектора $\vec{a} = \vec{a}(t)$ и тензора $\tensor{\Pi}=\tensor{\Pi}(t)$ справедливы следующие формулы:
	\begin{enumerate}	
		\item 
		$
			\displaystyle\dt{ (\tensor{\Pi} \cdot \vec{a})} = \dt{\tensor{\Pi}} \cdot \vec{a} + \tensor{\Pi} \cdot  \dt{\vec{a}}
		$;
		\item 
		$
		\displaystyle\dt{ (\tensor{\Pi} \times \vec{a})} = \dt{\tensor{\Pi}} \times \vec{a} + \tensor{\Pi} \times  \dt{\vec{a}}
		$;
		\item
		$
		\displaystyle\dt{(\tensor{\Pi}^{-1}) }= -\tensor{\Pi}^{-1} \cdot \dt{\tensor{\Pi}} \cdot \tensor{\Pi}^{-1}.
		$
	\end{enumerate}

	Первые два равенства следуют из определения производной от тензора, последнее -- из определения обратного тензора.
	}
	
}

\frame{
	\frametitle{ Тензорное поле }
	
	\begin{dfn}
		\parbox{\textwidth}{
			Если в каждой точке пространства определён тензор, то говорят что задано \alert{тензорное поле}
			\[
			\tensor{\Pi}=\tensor{\Pi}(\vec{r})= \vec{i}_1 \vec{p}_1(\vec{r}) + \vec{i}_2 \vec{p}_2(\vec{r})+  \vec{i}_3 \vec{p}_3(\vec{r}).
			\]
		}
		
	\end{dfn}\pause

	\begin{dfn}
	\parbox{\textwidth}{
		Назовём \alert{потоком тензора} через поверхность $S$ вектор, образованный по формуле
		\[
		\int\limits_S \vec{n}\cdot\tensor{\Pi} dS, где
		\]
		$\vec{n}$ -- вектор внешней единичной нормали к поверхности $S$.
	}
	
	\end{dfn}

}

\frame{
	\frametitle{Теорема о потоке тензора}
	
	\begin{theorems}
		\parbox{\textwidth}{
		Для любого объёма $V$, ограниченного поверхностью $S$, справедлива формула
		\[
			\int\limits_S \vec{n}\cdot\tensor{\Pi} dS = 
			\int\limits_V \left(
			\pd{\vec{p}_1}{x_1} + \pd{\vec{p}_2}{x_2} + \pd{\vec{p}_3}{x_3}
			\right) dV.
		\]
		}

	\end{theorems}
	
}


\end{document}

