\documentclass{beamer}

\usepackage{beamerthemesplit}
\usetheme{Singapore} %Copenhagen}

\input{../../include/preamble.inc} 
\input{../../include/definitions.inc} 
\input{../../include/author.inc} 

%\usepackage{textcomp}
%\usepackage{amssymb,amsmath}
%%\usepackage{animate}
%%\usepackage{longtable}
%\usepackage{xcolor}
%\usepackage{enumitem}
%
%\newcounter{N}

%% Форматирование окружения enumerate
%\usepackage{ragged2e}
%\let\olditem\item
%\renewcommand\item{\olditem\justifying}


\title[]{Сведения из теории матриц}

%\author[]{ {\em Верещагин Антон Сергеевич}
%	\\
%	канд. физ.-мат. наук, доцент\\
%	\bigskip
%	Кафедра аэрогидродинамики ФЛА НГТУ
%}

%\newtheorem{dfn}{Определение}  
%\newtheorem{theorems}{Теорема}  
%\newtheorem{property}{Свойство}  
%
%
%\newcommand{\Rn}{\mathrm{R}^n}
%\newcommand{\Sm}{\mathrm{S}^m}
%\newcommand{\Ql}{\mathrm{Q}^l}
%
%\newcommand{\Rd}[1]{\mathbb{R}^{#1}}
%\newcommand{\Vn}{\mathrm{V}^n}
%
%
%\newcommand{\oper}[1]{\mathcal{#1}}

\begin{document}

\frame{\titlepage}


\frame{
\frametitle{Аннотация}
\parbox{\textwidth}{
Сведения из теории матриц. Теоремы о разрешимости систем линейных уравнений. 
}
}

\frame{
\frametitle{Прямоугольные матрицы}

\begin{dfn}
\parbox{\textwidth}{
Прямоугольную  таблицу размера  $m \times n$ ($m$ строк, $n$ столбцов) из действительных чисел будем называть прямоугольной матрицей и обозначать:
\[
A=
\begin{pmatrix} 
a_{11} & a_{12} & \ldots & a_{1n} \\
a_{21} & a_{22} & \ldots & a_{2n} \\
\vdots & \vdots & \ddots & \vdots \\
a_{m1} & a_{m2} & \ldots & a_{mn}
\end{pmatrix}=
(a_{ik})_{1\leq i \leq m, 1\leq k \leq n}.
\]
}
\end{dfn}

}



\frame{
\frametitle{Определитель матрицы}
\begin{dfn}
Если $m=n$, тогда матрица $A$ называется квадратной.
\end{dfn}

\begin{dfn}
\parbox{\textwidth}{
Определителем квадратной матрицы $A=(a_{ik})_{1\leq i,k \leq n}$ называется следующая функция от элементов матрицы:
\[
|A| = \operatorname{det} A = \sum_{\alpha_1,\alpha_2,\ldots,\alpha_n}(-1)^{N(\alpha_1,\alpha_2,\ldots,\alpha_n)}
a_{1,\alpha_1}a_{2,\alpha_2}\ldots a_{n,\alpha_n},
\]
где  $\alpha _{1},\alpha _{2},\ldots ,\alpha _{n}$ -- перестановка чисел от $1$ до $n$, $N(\alpha _{1},\alpha _{2},\ldots ,\alpha _{n})$ -- число инверсий в перестановке, суммирование проводится по всем перестановкам порядка  $n$.
}
\end{dfn}
}

\frame{
\frametitle{Разложение по строке или столбцу}
В общем случае, определитель можно вычислить, применив следующую рекурсивную формулу:
\[
|A| =\sum _{j=1}^{n}(-1)^{1+j}a_{1j}{\bar {M}}_{j}^{1},
\]
где ${\bar {M}}_{j}^{1}$ -- дополнительный минор к элементу $a_{1j}$. Эта формула называется разложением по 1-ой строке.

\begin{dfn}
\parbox{\textwidth}{
Если $|A|=0$, то матрица называется {\em особенной}, иначе -- {\em неособенной}.
}
\end{dfn}

}

\frame{
\frametitle{Свойства определителя}


\begin{property}
\begin{enumerate}
\item Определитель матрицы не меняется при транспонировании матрицы.
\item Если две строки (столбца) матрицы совпадают, то её определитель равен нулю. 
\item При добавлении к любой строке (столбцу) линейной комбинации других строк (столбцов) определитель не изменится. 
\item Если хотя бы одна строка (столбец) матрицы нулевая, то определитель равен нулю.
\item Если переставить две строки (столбца) матрицы, то её определитель умножается на (-1).
\end{enumerate}



\end{property}



}

\frame{
\frametitle{Минор матрицы}

\begin{dfn}
\parbox{\textwidth}{
Из элементов прямоугольной матрицы  $A$ размера $m\times n$ можно составить квадратную матрицу размера $p\times p$, выбрав из неё строки с номерами $i_1, i_2,\ldots,i_p$ и столбцы с номерами $k_1, k_2,\ldots, k_p$. Определитель полученной матрицы называется {\em минором $p$-ого порядка}:
\[
A
\begin{pmatrix}
i_1 & i_2 & \ldots & i_p \\
k_1 & k_2 & \ldots & k_p \\
\end{pmatrix}=
\begin{vmatrix}
a_{i_1k_1} & a_{i_1k_2} & \ldots & a_{i_1k_p} \\
a_{i_2k_1} & a_{i_2k_2} & \ldots & a_{i_2k_p} \\
\vdots & \vdots & \ddots & \vdots \\
a_{i_pk_1} & a_{i_pk_2} & \ldots & a_{i_pk_p}
\end{vmatrix}.
\]
}
\end{dfn}

\begin{dfn}
\parbox{\textwidth}{
Если $i_1=k_1$, $i_2=k_2$, \ldots, $i_p=k_p$, то минор называется главным.
}
\end{dfn}
}


\frame{
\frametitle{Ранг матрицы}
\begin{dfn}
\parbox{\textwidth}{
Наибольший из порядков  отличных от нуля миноров называется рангом матрицы $A$ и обозначается $r_A$.
}
\end{dfn}
\begin{property}
\parbox{\textwidth}{
\begin{enumerate}
\item Ранг прямоугольной $m \times n$ матрицы $A$ меньше либо равен $\min(m,n)$: $r_A \leq \min(m,n)$.
\item Если у квадратной $n\times n$ матрицы $A$ определитель равен $0$, тогда $r_A < n$, в противном случае $r_A = n$.
\end{enumerate}

}
\end{property}

}

\frame{
\frametitle{ Произведение матриц }
\begin{dfn}
\parbox{\textwidth}{
Произведением $m \times s$ матрицы  $A$ на $s \times n$ матрицу $B$ называется $m\times n$ матрица $C$, элементы которой определяются по формулам:
\[
c_{ij}=\sum_{k=1}^s a_{ik}b_{kj},\quad
(1 \leq i \leq m, 1\leq j \leq n).
\]
}
\end{dfn}
}

\frame{
\frametitle{ Теоремы о ранге }
\begin{theorems}
\parbox{\textwidth}{
 Ранг произведения двух матриц не превосходит ранги сомножителей, т.е. если $C=AB$, то $r_C\leq \min(r_A,r_B)$.
}
\end{theorems}

\begin{theorems}
\parbox{\textwidth}{
 При умножении прямоугольной матрицы слева или справа на неособенную матрицу ранг исходной матрицы не изменяется.
}
\end{theorems}
}

\frame{
\frametitle{ Обратная матрица }
\begin{dfn}
\parbox{\textwidth}{
Единичной  матрицей $E$ называют диагональную $n\times n$ матрицу с единицами на главной диагонали:
\[
E = 
\begin{pmatrix} 
1 & 0 & \ldots & 0 \\
0 & 1 & \ldots & 0 \\
\vdots & \vdots & \ddots & \vdots \\
0 & 0 & \ldots &1
\end{pmatrix}
\]
}
\end{dfn}
\begin{dfn}
\parbox{\textwidth}{
Если для квадратной $n\times n$ матрицы $A$ существует матрица $B$ такая, что 
\[
A B = BA = E,
\]
тогда $B$ называют матрицей обратной к $A$ и обозначают $A^{-1}$.
}
\end{dfn}
}

\frame{
\frametitle{ Некоторые свойства операций над матрицами }
\begin{property}
\parbox{\textwidth}{
\begin{enumerate}
\item В общем случае, вообще говоря, $AB \neq BA$.
\item Для любой квадратной матрицы $A$ и соответствующего размера единичной $E$: $AE$=$EA$ =$A$.
\item $|AB|=|A||B|$.
\item $|A^{-1}|=|A|^{-1}$.
\item Для квадратной матрицы $A$ существует обратная матрица $A^{-1}$ тогда и только тогда, когда $|A|\neq 0$.
\end{enumerate}
}
\end{property}
}

\frame{
\frametitle{ Cистемы линейных уравнений }
\parbox{\textwidth}{
Систему из $m$ линейных уравнений c $n$ переменными 
\begin{eqnarray*}
a_{11}x_1 + a_{12} x_2+ \ldots + a_{1n} x_n & = & b_1, \\
a_{21}x_1 + a_{22} x_2+ \ldots + a_{2n} x_n & = & b_1, \\
\ldots\ldots\ldots & \ldots & \ldots \\
a_{m1}x_1 + a_{m2} x_2+ \ldots + a_{mn} x_n & = & b_m \\
\end{eqnarray*}
можно переписать используя $m\times n$ матрицу $A=(a_{ij})_{1\leq i \leq m, 1\leq j \leq n}$, $n$ вектор столбец $x=(x_j)_{1\leq j \leq n}$, $m$ вектор столбец $b = (b_i)_{1\leq i \leq m}$ в кратком матричном виде
\[
Ax=b.
\]
}
}

\frame{
\frametitle{ Теоремы о единственности решения однородной системы линейных уравнений }
\parbox{\textwidth}{
\begin{theorems}
\parbox{\textwidth}{
Пусть $A$ квадратная $n\times n$ матрица, $x$ столбец размера $n$. Уравнение $Ax=0$ имеет ненулевое решение тогда и только тогда, когда матрица $A$ особенная ($|A|=0$), иначе говоря, когда ранг матрицы $A$ меньше числа неизвестных $(r_A<n)$. В противном случае неособенной матрицы $(|A|\neq 0)$, иначе говоря, $r_A=n$) оно имеет только нулевое решение.
}
\end{theorems}

\begin{theorems}
\parbox{\textwidth}{
 Пусть $A$ прямоугольная $m\times n$ матрица, $x$ столбец размера $n$. Система линейных уравнений $Ax=0$ имеет ненулевое решение тогда и только тогда, когда ранг матрицы $A$ меньше числа неизвестных ($r_A < n$). В противном случае $(r_A=n)$ она имеет только нулевое решение.
}
\end{theorems}
}
}


\frame{
	\frametitle{ Теорема о разрешимости системы линейных уравнений }
	
	\begin{exampleblock}{Теорема Кронекера-Капелли}
		\parbox{\textwidth}{
		Система линейных уравнений $Ax=b$ разрешима тогда и только тогда, когда ранг матрицы $A$ равен рангу матрицы $A|b$, где $A|b$~-- расширенная матрица, полученная из матрицы $A$ приписыванием столбца $b$. 	
		}
	\end{exampleblock}
}

\frame{
	\frametitle{Литература}
	
	\begin{literature}
		\item 
		{\em Курош~А.~Г.} Курс высшей алгебры. Учебник. 17-е изд., стер. -- СПб.: Издательство <<Лань>>, 2008.
		
		\item {\em Ветлуцкий~В.~Н.} Специальные разделы высшей математики: учеб. пособие. -- Новосибирск: Изд-во НГТУ, 2005.
	\end{literature}
	
}


\end{document}
