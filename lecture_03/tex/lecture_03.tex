\documentclass{beamer}

\usepackage{beamerthemesplit}
\usetheme{Singapore} %Copenhagen}
%\usecolortheme{whale}

\usepackage[T2A]{fontenc}
\usepackage[utf8]{inputenc}
\usepackage[russian]{babel}


\usepackage{textcomp}
\usepackage{amssymb,amsmath}
%\usepackage{animate}
%\usepackage{longtable}
\usepackage{xcolor}

\newcounter{N}

%% Форматирование окружения itemize
%\usepackage{ragged2e}
%\let\olditem\item
%\renewcommand\item{\olditem\justifying}


\title[]{Метрика и норма в векторном пространстве. Квадратичные формы.}

\author[]{ {\em Верещагин Антон Сергеевич}
	\\
	канд. физ.-мат. наук, доцент\\
	\bigskip
	Кафедра аэрогидродинамики ФЛА НГТУ
}
%



\newtheorem{dfn}{Определение}  
\newtheorem{theorems}{Теорема}  

\newcommand{\Rn}{\mathrm{R}^n}
\newcommand{\Sm}{\mathrm{S}^m}
\newcommand{\Ql}{\mathrm{Q}^l}

\newcommand{\Rd}[1]{\mathbb{R}^{#1}}
\newcommand{\Vn}{\mathrm{V}^n}


\newcommand{\oper}[1]{\mathcal{#1}}

\begin{document}

\frame{\titlepage}


\frame{
\frametitle{Аннотация}

\parbox{\textwidth}{
Метрические пространства. Нормы вектора и оператора. Квадратичные формы. Приведение формы к сумме квадратов. Условие положительной определенности квадратичной формы.
}
}


\frame{
\frametitle{Расстояние между векторами}

\begin{dfn}
\parbox{\textwidth}{
Пусть $\Rn$ -- $n$-мерное векторное пространство. \pause \alert{Расстоянием} между векторами $\vec{x}$ и $\vec{y}$ называется неотрицательная скалярная функция $\rho(\vec{x}, \vec{y}) \geq 0$,  \pause удовлетворяющая аксиомам:
\begin{itemize}
\item $\rho(\vec{x},\vec{y}) = 0$ тогда и только тогда, когда $\vec{x}=\vec{y}$; \pause 
\item $\rho(\vec{x},\vec{y}) = \rho(\vec{y},\vec{x})$ (аксиома симметрии); \pause 
\item $\rho(\vec{x},\vec{z}) \leq \rho(\vec{x},\vec{y}) + \rho(\vec{y},\vec{z})$ (аксиома треугольника). \pause 
\end{itemize}
}
\end{dfn}

\bigskip
\fbox{
\parbox{\textwidth}{
Если в пространстве определено расстояние между векторами, то говорят, что определена \alert{\it метрика}.
}
}
}

\frame{
\frametitle{Метрические пространства}

\begin{dfn}
\parbox{\textwidth}{
Векторное пространство $\Rn$ с введенной метрикой называется \alert{\it метрическим пространством}.
}
\end{dfn} \pause 

\bigskip
\begin{exampleblock}{Примеры метрик в $\Rn$:}
\parbox{\textwidth}{
\begin{itemize}
\item $\rho(\vec{x},\vec{y}) = \sum\limits_{i=1}^{n}|x_i-y_i|$, \pause 
\item $\rho(\vec{x},\vec{y}) = \sqrt{\sum\limits_{i=1}^{n}(x_i-y_i)^2}$.
\end{itemize}
}
\end{exampleblock}

}


\frame{
\frametitle{Норма вектора}
\begin{dfn}
\parbox{\textwidth}{
\alert{Нормой вектора} $||\vec{x}||$ в $n$-мерном векторном пространстве  \pause называется неотрицательная скалярная функция от вектора $\vec{x}$,  \pause удовлетворяющая следующим аксиомам:  \pause 
\begin{itemize}
\item $||\vec{x}||=0$ тогда и только тогда, когда $\vec{x}=0$; \pause 
\item $||\alpha \vec{x}||=|\alpha|||\vec{x}||$, где $\alpha$ -- произвольное число; \pause 
\item $||\vec{x}+\vec{y}||\leq||\vec{x}||+||\vec{y}||$ (неравенство треугольника).
\end{itemize}
}
\end{dfn}


}

\frame{
\frametitle{Нормированные пространства}

\begin{dfn}
\parbox{\textwidth}{
Векторное пространство $\Rn$ с введенной на нем нормой называется \alert{\it нормированным векторным пространством}.
}
\end{dfn}  \pause 

\bigskip
\begin{exampleblock}{Примеры норм в $\Rn$:}
\begin{columns}
\begin{column}{0.4\textwidth}
\begin{itemize}
\item $||\vec{x}|| = \sum\limits_{i=1}^{n}|x_i|$, \pause 
\item $||\vec{x}|| = \sqrt{\sum\limits_{i=1}^{n}x_i^2}$. \pause 
\end{itemize}
\end{column}

\begin{column}{0.6\textwidth}
%\parbox{\textwidth}{
%Если в векторном пространстве введена метрика $\rho(\vec{x},\vec{y})$, тогда можно ввести норму по формуле:
%\begin{itemize}
%\item $||\vec{x}|| = \rho(\vec{x}, 0)$.
%\end{itemize}
%
%\medskip
%(Докажите самостоятельно.)
%}
\end{column}


\end{columns}


\end{exampleblock}
}

\frame{
\frametitle{Норма линейного оператора}

\begin{dfn}
\parbox{\textwidth}{
\alert{Нормой линейного оператора} $\oper{A}$ в нормированном пространстве $\Rn$ \pause  называется наименьшее из чисел $C$ таких, \pause что $\forall \vec{x}\in \Rn$
\[
||\oper{A}\vec{x}|| \leq C ||\vec{x}|| \pause 
\]
и обозначается $||\oper{A}||$,  \pause или, по-другому,
\[
||\oper{A}||=\inf \{C : \forall \vec{x}\in \Rn \quad ||\oper{A}\vec{x}|| \leq C ||\vec{x}|| \}.
\]

}
\end{dfn}

}

\frame{
\frametitle{Связь между различными определениями нормы линейного оператора}
Пусть $\oper{A}$ -- линейный оператор в $\Rn$,  \pause $\vec{x}$ -- произвольный вектор из $\Rn$.  \pause Рассмотрим
\[
\frac{||\oper{A}\vec{x}||}{||\vec{x}||}= \pause 
||\frac{1}{||\vec{x}||}\oper{A}\vec{x}||= \pause 
\left|\left|\oper{A}\left(\frac{\vec{x}}{||\vec{x}||}\right) \right|\right| =  \pause 
||\oper{A}\vec{y}|| \leq C,
\]
где $\vec{y}=\vec{x}/||\vec{x}||$,  \pause причем
\[
||\vec{y}|| =  \pause 
\left|\left| \frac{\vec{x}}{||\vec{x}||}  \right|\right| =  \pause 
\frac{||\vec{x}||}{||\vec{x}||} =  \pause 1. \pause 
\]
Таким образом,  \pause $\vec{y}\in\Rn$ -- произвольный вектор,  \pause такой что $||\vec{y}||=1$.
}

\frame{
\frametitle{Норма линейного оператора}

\begin{dfn}[альтернативное определение]
\parbox{\textwidth}{
\alert{Нормой линейного оператора} $\oper{A}$  \pause в нормированном пространстве $\Rn$  \pause называется наибольшее значение,  \pause принимаемое функцией 
$||\oper{A}\vec{x}||$  $\forall \vec{x}\in R^n$,  \pause таких, что $||\vec{x}||=1$,  \pause или, по-другому, \pause 
\[
||\oper{A}||= \pause \sup \{||\oper{A}\vec{x}|| : \forall \vec{x}\in \Rn \quad ||\vec{x}||=1\}.
\]
}
\end{dfn}


}


\frame{
\frametitle{Пример: проекция вектора на плоскость}

\parbox{\textwidth}{
Рассмотрите оператор проектирования вектора на плоскость и посчитайте норму такого линейного оператора.
}

}

\frame{
\frametitle{Определение квадратичной формы}
\begin{dfn}
\parbox{\textwidth}{
\alert{Квадратичной формой} называется однородный многочлен второй степени  \pause  относительно $n$ переменных $x_1$, $x_2$, ... , $x_n$: \pause 
\[
A(x,x)=\sum\limits_{i,k=1}^n a_{ik}x_ix_k \quad  (a_{ik}=a_{ki}).
\]
}
\end{dfn} \pause 

\begin{exampleblock}{Примеры:}
\begin{itemize}
\item $n=1$: $A_1(x,x) = a_{11} x_1^2$, \pause 
\item $n=2$: $A_2(x,x) = a_{11}x_1^2 + 2 a_{12} x_1 x_2 + a_{22}x_2^2$, \pause 
\item $n=3$: $A_3(x,x) = x_1^2 + 2 x_1 x_2 + x_2^2 + x_2 x_3$. \pause 
\item $n=3$: $A_4(x,x) = x_1^2 + x_2^2 - x_3^2$.
\end{itemize}
\end{exampleblock}


}


\frame{
\frametitle{Матрица, соответствующая квадратичной форме}

\parbox{\textwidth}{
Из коэффициентов квадратичной формы  \pause можно составить квадратную симметричную матрицу $A$: \pause 
\[
A =  \left(
\begin{array}{cccc}
a_{11} & a_{12} & \ldots & a_{1n} \\
a_{12} & a_{22} & \ldots & a_{2n} \\
\vdots & \vdots & \ddots & \vdots\\
a_{1n} & a_{2n} & \ldots & a_{nn} \\
\end{array}\right),\quad
A=A^{\rm t}.
\] \pause 

C помощью матрицы $A$ квадратичную форму $A(x,x)$ \pause  можно переписать в виде:
\[
A(x,x)=x^{\rm t}Ax,
\] \pause 
где $x=\{x_1, x_2,\ldots,x_n\}^{\rm t}$ -- вектор столбец из переменных $x_i$ ($i=\overline{1,n}$).

}


}

\frame{
\frametitle{Вещественная квадратичная форма}
\begin{dfn}
\parbox{\textwidth}{
Если матрица $A$ есть вещественная симметричная матрица,  \pause то соответствующая ей квадратичная форма называется \alert{вещественной формой}. 
}
\end{dfn} \pause 

\begin{dfn}
\parbox{\textwidth}{
Определитель матрицы $|A|$  \pause называется \alert{дискриминантом} квадратичной формы.  \pause Если $|A|=0$, то квадратичная форма \alert{сингулярна},  \pause иначе \alert{регулярна}.
}
\end{dfn} \pause 

\begin{dfn}
\parbox{\textwidth}{
Ранг матрицы $A$, отвечающей квадратичной форме,  \pause есть \alert{ранг квадратичной формы}.
}
\end{dfn}

}

\frame{
\frametitle{Замена переменных}
\parbox{\textwidth}{
Рассмотрим линейную замену переменных \pause 
\[
x_i=\sum\limits_{k=1}^n t_{ik}\xi_k\quad (i=1,2,\ldots,n),
\] \pause 
что в матричном форме можно записать как \pause 
\[
\begin{array}{c}
x = T \xi,\\ \pause 
\\
x=\{x_1, x_2,\ldots,x_n\}^{\rm t},\quad
\xi=\{\xi_1, \xi_2,\ldots,\xi_n\}^{\rm t},\\ \pause 
T=(t_{ij})_{1\leq i,j \leq n},\quad
|T|\neq 0. 
\end{array}
\]  \pause 
Тогда квадратичная форма будет иметь вид: \pause 
\[
A(x,x)= \pause 
x^{\rm t}Ax= \pause 
(T \xi)^{\rm t} A T \xi=  \pause 
\xi^{\rm t}\left( T^{\rm t}AT \right)\xi= \pause 
\xi^{\rm t}\hat{A}\xi=\hat{A}(\xi,\xi),  \pause 
\]
где $\hat{A}=T^{\rm t}AT$.
}

}


\frame{
\frametitle{Свойства $\hat{A}$}
\begin{exampleblock}{Симметричность $\hat{A}$:}
\parbox{\textwidth}{ \pause 
\[
\hat{A}^{\rm t}= \pause (T^{\rm t}AT)^{\rm t}=  \pause T^{\rm t} A T = \hat{A}. \pause 
\]
}
\end{exampleblock}
\begin{dfn}
\parbox{\textwidth}{
Две симметрические матрицы $A$ и $\hat{A}$, \pause  связанные равенством 
\[
\hat{A}=T^{\rm t}AT,
\]  \pause 
называются \alert{конгруэнтными},  \pause где $T$ -- неособенная матрица.
}
\end{dfn} \pause 

\begin{exampleblock}{Ранг $\hat{A}$:}
Так как $|T|\neq 0$,  \pause то по теореме о ранге  \pause ранг $A$ равен рангу матрицы $\hat{A}$.
\end{exampleblock}

}

\frame{
\frametitle{Ранг диагональной матрицы}
\begin{theorems}
\normalfont
\parbox{\textwidth}{
Ранг диагональной матрицы  \pause равен количеству ненулевых элементов на диагонали.
}
\end{theorems} \pause 

\begin{proof}
\parbox{\textwidth}{
Рассмотрим диагональную матрицу $D$.  \pause Пусть в матрице $r$ ненулевых диагональных элементов $a_{i_1}$, $a_{i_2}$,\ldots, $a_{i_r}$.  \pause Очевидно, что главный минор $M\neq 0$, где \pause 
\[
D = \left(
\begin{array}{cccc}
a_1 & 0 & \cdots & 0\\
0  & a_2 & \cdots & 0\\
\vdots &  \vdots & \ddots & \vdots \\
0 & 0 & \cdots & a_n
\end{array}
\right),\quad \pause 
M=D\left(
\begin{array}{cccc}
i_1 & i_2 & \cdots & i_r\\
i_1 & i_2 & \cdots & i_r\\
\end{array}
\right)
\neq 0.
\] \pause 
Все остальные миноры большего размера будут равны $0$,  \pause т.к. определитель будет считаться от треугольной матрицы с нулями на диагонали. \pause  Следовательно, ранг матрицы $D$ равен $r$.
}
\end{proof}

}

\frame{
\frametitle{О представлении квадратичной формы в виде суммы квадратов}

\begin{theorems}
\parbox{\textwidth}{\normalfont
Любую квадратичную форму  \pause с помощью линейной замены переменных  \pause можно привести к виду
\[
A(x,x)=\sum\limits_{i=1}^r a_iX_i^2,
\] \pause 
где $X_i=\sum\limits_{k=1}^n\alpha_{ik}x_k$ и $r\leq n$.
}
\end{theorems}  

}

\frame{
\frametitle{О представлении квадратичной формы в виде суммы квадратов}
\begin{proof}

\only<1-12>{
\parbox{\textwidth}{
По теореме о собственных значениях симметричной матрицы 
\[
A=QDQ^{-1},
\] \pause 
где $D$ -- диагональная матрица, составленная из собственных значений $A$;  \pause $Q$ -- ортогональная матрица  \pause ($Q^{\rm t}=Q^{-1}$) из собственных векторов, \pause  образующих ортонормированный базис.

\medskip
Это означает, что если в качестве матрицы для замены переменных положить $T=Q$,  \pause тогда
\[
\hat{A}= \pause 
T^{\rm t}AT= \pause 
Q^{\rm t} A Q =  \pause 
Q^{-1} QDQ^{-1} Q =  \pause 
D. \pause 
\]
Пусть $r=r(D)=r(A)$  \pause и равно количеству ненулевых элементов на диагонали матрицы $D$.
}


}



\only<13>{
\parbox{\textwidth}{

Рассмотрим
\[
\begin{array}{l}
\hat{A}(\xi,\xi) = \xi^{\rm t} D \xi=\\
=(\xi_1,\dots,\xi_r, \xi_{r+1}, \ldots ,\xi_n)
\left(
\begin{array}{cccccc}
\lambda_1 & \ldots & 0 & 0 & \ldots  & 0 \\
\vdots & \ddots & \vdots & \vdots & & \vdots \\
0 & \ldots & \lambda_r & 0 & \ldots  & 0 \\
0 & \ldots & 0 & 0 & \ldots  & 0 \\
\vdots &  & \vdots & \vdots & \ddots & \vdots \\
0 & \ldots & 0 & 0 & \ldots  & 0 
\end{array}
\right)
\left(
\begin{array}{c}
\xi_1\\
\vdots \\
\xi_r\\
\xi_{r+1}\\
\vdots \\
\xi_n
\end{array}
\right)=\\
\\
\multicolumn{1}{c}{
=\lambda_1\xi_1^2 + \lambda_2\xi_2^2+\ldots+\lambda_r\xi_r^2.}
\end{array}
\]
}
}
\alt<13>{\qedhere}{\phantom\qedhere}
\end{proof}

}


\frame{
\frametitle{Закон инерции квадратичных форм}
\begin{theorems}[закон инерции квадратичных форм]
\normalfont
\parbox{\textwidth}{
При представлении квадратичной формы в виде суммы независимых квадратов,  \pause указанном выше, число положительных и отрицательных коэффициентов перед квадратами \pause  переменных не зависит от представления.
}
\end{theorems}
}

\frame{
\frametitle{Закон инерции квадратичных форм}
\begin{proof}
\only<1-7>{
\parbox{\textwidth}{
Пусть $A(x,x)$ -- квадратичная форма,  \pause приведена к сумме квадратов с помощью двух преобразований $x=T\xi$, $x=Q\eta$: \pause 
\[
A(\xi,\xi) = \sum\limits_{i=1}^r a_i \xi_i^2,\quad
A(\eta,\eta) = \sum\limits_{i=1}^r b_i \eta_i^2,
\] \pause 
где $Q$, $T$ -- квадратные неособенные матрицы. \pause 
Пусть среди коэффициентов $a_i$ и $b_j$  \pause различное число положительных и отрицательных слагаемых: \pause 
\[
\begin{array}{c}
\overbrace{a_1, a_2,\ldots, a_h}^{>0}, \overbrace{a_{h+1},\ldots, a_g, a_{g+1},\ldots, a_{r}}^{<0}\\
\underbrace{b_1, b_2,\ldots, b_h, b_{h+1},\ldots,b_g}_{>0}, \underbrace{b_{g+1},\ldots, b_{r}}_{<0}\\
\end{array}\quad
(h<g\leq r).
\]
}
}
\only<8-11>{
\parbox{\textwidth}{
Рассмотрим равенство 
\[
A(x,x)=\overbrace{\sum\limits_{i=1}^h a_i \xi_i^2}^{\geq 0}+\overbrace{\sum\limits_{i=h+1}^r a_i \xi_i^2}^{\leq 0}=
\overbrace{\sum\limits_{i=1}^g b_i \eta_i^2}^{\geq 0}+
\overbrace{\sum\limits_{i=g+1}^r b_i \eta_i^2}^{\leq 0},  
\]
}
\only<9-11>{ 
где $x=T\xi=Q\eta$.
}
\only<10-11>{ 
Выберем такой набор $x_j$, чтобы
\begin{center}
$\xi_1=\xi_2=\ldots=\xi_h=0,\quad \eta_{g+1}=\eta_{g+2}=\ldots=\eta_{r}=0$,\\
$\xi_j\neq 0$ для некоторого $j$ ($h<j\leq r$).
\end{center}
}
\only<11>{ 
Тогда в верхнем равенстве с одной стороны получится, что $A(x,x) < 0$, а с другой $A(x,x)\geq 0$, чего не может быть, значит $h=g$.
}
}
\only<12->{
\parbox{\textwidth}{
Рассмотрим систему из $n$ линейных уравнений от $2n$ неизвестных $\xi_i$ и $\eta_i$ ($i=\overline{1,n}$) 
}
}
\only<13->{
\[
T\xi-Q\eta=0
\]
и $h+r-g$ дополнительных линейных соотношений
\begin{center}
$\xi_1=\xi_2=\ldots=\xi_h=0,\quad \eta_{g+1}=\eta_{g+2}=\ldots=\eta_{r}=0$.
\end{center}
}
\only<14->{
\parbox{\textwidth}{
Так как $r-(g-h) < n$, то ранг матрицы суммарной системы с дополнительными соотношениями меньше числа неизвестных, поэтому существует нетривиальное решение, позволяющее считать $\xi_j\neq 0$ для некоторого $j$ ($h<j\leq r$).
}
}
\alt<3>{\qedhere}{\phantom\qedhere}
\end{proof}
}


\frame{
\frametitle{Метод выделения квадратов Лагранжа}
\parbox{\textwidth}{
При выделении квадратов в квадратичной форм возможны два случая: \pause
\begin{enumerate}
\item Для некоторого $g\leq n$ $a_{gg}\neq 0$, тогда исключаем $x_g$ по формуле \pause
\[
A(x,x)=\frac{1}{a_{gg}}\left(\sum\limits_{k=1}^na_{gk}x_k\right)^2+A_1(x,x).  \pause
\]
\item Пусть $a_{gg}=$ и $a_{hh}=0$, но $a_{gh}=a_{hg}\neq 0$,  \pause
тогда
\[
A(x,x)=\frac{1}{2a_{hg}}\left[\sum\limits_{k=1}^n (a_{gk}+a_{hk})x_k\right]^2-\frac{1}{2a_{hg}}\left[\sum\limits_{k=1}^n (a_{gk}-a_{hk})x_k\right]^2+
\]
\[
+A_2(x,x).
\]

\end{enumerate}
}
}

\frame{
\frametitle{Теорема Якоби}
\begin{theorems}[Теорема Якоби]
\normalfont\scriptsize
\parbox{\textwidth}{
Пусть квадратичная форма имеет вид   \pause 
\[
A(x,x)=\sum\limits_{i,k=1}^n a_{ik}x_ix_k \quad  (a_{ik}=a_{ki}),
\] \pause 
и пусть главные миноры соответствующей ей матрицы $A$ \pause 
\[
D_1=a_{11},\quad   \pause 
D_2=
\begin{pmatrix}
1 & 2\\
1 & 2 \\
\end{pmatrix}, \pause 
\ldots,
D_n=
\begin{pmatrix}
1 & 2 & \ldots & n\\
1 & 2 & \ldots & n\\
\end{pmatrix}, \pause 
\] 
все отличны от нуля.  \pause Тогда существуют линейные формы $\xi_i=\sum\limits_{k=1}^n\alpha_{ik}x_k$, \pause 
для которых квадратичная форма запишется в виде  \pause 
\[
A(\xi,\xi)=\frac{D_0}{D_1}\xi_1^2+\frac{D_1}{D_2}\xi_2^2+\ldots+\frac{D_{n-1}}{D_n}\xi_n^2, \quad D_0=1.
\]
(без доказательства)
}
\end{theorems}
}


\frame{
\frametitle{Положительно определенные квадратичные формы}
\begin{dfn}
\parbox{\textwidth}{
Вещественная квадратичная форма $A(x,x)$ называется \alert{положительно определенной},  \pause если для любых значений переменных $x_1, \ldots ,x_n$,  \pause не равных одновременно $0$, \pause  она принимает только положительные значения. \pause 
}
\end{dfn}

\begin{theorems}
\normalfont
\parbox{\textwidth}{
Если положительно определенная квадратичная форма приведена к сумме квадратов,  \pause то все коэффициенты перед квадратами всегда больше $ 0 $  \pause и количество квадратов равно количеству переменных.
}
\end{theorems}

}

\frame{
\frametitle{Представление квадратичной формы в виде суммы квадратов}
\begin{proof}
\parbox{\textwidth}{
Предположим, положительно определенная квадратичная форма $A(x,x)$ приведена к сумме квадратов  \pause с помощью линейного преобразования $x=T\xi$,  \pause тогда
\[
A(x,x)=A(\xi,\xi)=\sum\limits_{i=1}^n a_{i} \xi_i^2.
\] \pause 
Предположим, что  $a_j \leq 0$ при некотором $j$.  \pause Тогда рассмотрим вектор $\xi^0=\{0,\ldots,0,\underbrace{1}_{\text{\tiny j-е место}},0,\ldots,0\}$  \pause и  $x^0=T\xi^0\neq 0$.  \pause С одной стороны $A(x^0,x^0)>0$,  т.к. форма положительно определена,  \pause с другой $A(\xi^0,\xi^0)\leq 0$ из за выбора $\xi_0$.
}
\end{proof}
}

\frame{
\frametitle{Критерий Сильвестера}
\begin{theorems}[критерий Сильвестера]
\normalfont
\parbox{\textwidth}{
Для того чтобы квадратичная форма
\[
A(x,x)=\sum\limits_{i,k=1}^na_{ik}x_ix_k
\] \pause 
была положительно определенной,  \pause необходимо и достаточно, чтобы все главные миноры формы были положительные.\\
(Доказательство вытекает из теоремы Якоби.)
}
\end{theorems}
}


\end{document}

%\begin{columns}и
%\begin{column}{0.5\textwidth}
%\end{column}
%
%\begin{column}{0.5\textwidth}
%\end{column}
%
%
%\end{columns}
