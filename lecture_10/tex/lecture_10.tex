\documentclass{beamer}

\usepackage{beamerthemesplit}
\usetheme{Singapore} %Copenhagen}
%\usecolortheme{whale}

\input{../../include/preamble.inc} 
\input{../../include/definitions.inc} 
\input{../../include/author.inc} 


\title[]{Операции с тензорами}

\begin{document}

\frame[plain]{\titlepage}


\frame[plain]{
\frametitle{Аннотация}
\parbox{\textwidth}{
Скалярное и векторное умножение тензора на вектор. Скалярное произведение тензоров.
}
}



\frame{
\frametitle{Скалярное и векторное умножение тензора на вектор}

\begin{dfn}
\parbox{\textwidth}{
Под \alert{скалярным произведением тензора} $ \tensor{\Pi} = \vec{i}_1\vec{p}_1+\vec{i}_2\vec{p}_2+\vec{i}_3\vec{p}_3$ \alert{на вектор} $ \vec{a}=\vec{i}_1a_1+\vec{i}_2a_2+\vec{i}_3a_3 $ \alert{справа} будем понимать вектор $ \vec{a'}'$:
\[ 
\vec{a'}=\tensor{\Pi}\cdot\vec{a}=\vec{i}_1(\vec{p}_1\cdot\vec{a})+\vec{i}_2(\vec{p}_2\cdot\vec{a})+\vec{i}_3(\vec{p}_3\cdot\vec{a}).
\]
%\[
%=\vec{i}_1(p_{11}a_1+p_{12}a_2+p_{13}a_3)+\vec{i}_2(p_{21}a_1+p_{22}a_2+p_{23}a_3)+\vec{i}_3(p_{31}a_1+p_{32}a_2+p_{33}a_3).
%\]
}
\end{dfn}
 \pause 

\begin{dfn}
\parbox{\textwidth}{
Под \alert{скалярным произведением вектора} $ \vec{a} $ \alert{на тензор} $ \tensor{\Pi} $ \alert{слева} понимается вектор $ \vec{a''} $:
\[ 
\vec{a''}=\vec{a}\cdot\tensor{\Pi}=(\vec{a}\cdot\vec{i}_1)\vec{p}_1+(\vec{a}\cdot\vec{i}_2)\vec{p}_2+(\vec{a}\cdot\vec{i}_3)\vec{p}_3=
\]
\[ 
=a_1\vec{p}_1+a_2\vec{p}_2+a_3\vec{p}_3.
\]
}
\end{dfn}
}

\frame{
\frametitle{Диада (повтор)}

\begin{dfn}
\parbox{\textwidth}{
Пусть $\vec{a}=\basis{i}_1a_1+\basis{i}_2a_2+\basis{i}_3a_3$ и $\vec{b}=\basis{i}_1b_1+\basis{i}_2b_2+\basis{i}_3b_3$, тогда \alert{диадным} или \alert{тензорными произведением} векторов $\vec{a}$ и $\vec{b}$ называется тензор, определяемый следующим соотношением: 
\[
\vec{a}\otimes\vec{b}=
\vec{a}\vec{b}= 
\begin{pmatrix}
a_1b_1 & a_1b_2 & a_1b_3 \\
a_2b_1 & a_2b_2 & a_2b_3 \\
a_3b_1 & a_3b_2 & a_3b_3 
\end{pmatrix}=
\basis{i}_1(a_1\vec{b})+\basis{i}_2(a_2\vec{b})+\basis{i}_3(a_3\vec{b}).
\]
}
\end{dfn} \pause 

\begin{exampleblock}{Линейность диады по каждому аргументу}
\[
(\vec{a}+\vec{b})\vec{c} = \vec{a}\vec{c}+ \vec{b}\vec{c}.
\]
\[
\vec{c}(\vec{a}+\vec{b}) = \vec{c}\vec{a}+ \vec{c}\vec{b}.
\]

\end{exampleblock}

}


\frame{
\frametitle{Скалярное произведение диады на вектор}

Пусть $\vec{a}$, $\vec{b}$, $\vec{c}$ -- вектора.

\[ 
(\vec{b}\vec{c})\cdot\vec{a}= \pause 
(\basis{i}_1b_1\vec{c}+\basis{i}_2b_2\vec{c}+\basis{i}_3b_3\vec{c})\cdot\vec{a}= \pause 
\basis{i}_1b_1(\vec{c}\cdot\vec{a})+\basis{i}_2b_2(\vec{c}\cdot\vec{a})+\basis{i}_3b_3(\vec{c}\cdot\vec{a})= \pause 
\]
\[
=
(\basis{i}_1b_1+\basis{i}_2b_2+\basis{i}_3b_3)(\vec{c}\cdot\vec{a})= \pause 
\vec{b}(\vec{c}\cdot\vec{a}).
\]

\medskip
\[ 
\vec{a}\cdot(\vec{b}\vec{c})= \pause 
\vec{a}\cdot(\basis{i}_1b_1\vec{c}+\basis{i}_2b_2\vec{c}+\basis{i}_3b_3\vec{c})= \pause 
a_1b_1 \vec{c} + a_2b_2 \vec{c} + a_3b_3 \vec{c}= \pause 
\]
\[
=
(a_1b_1 + a_2b_2 + a_3b_3) \vec{c}= \pause 
(\vec{a}\cdot\vec{b})\vec{c}.
\]
}

\frame{
\frametitle{Векторное произведение тензора на вектор}

\begin{dfn}
\parbox{\textwidth}{
Под \alert{векторным произведением тензора} $ \tensor{\Pi} $ \alert{на вектор} $\vec{a}$ \alert{справа} понимается новый тензор $ \tensor{\Pi'} $, вычисленный по формуле:
\[ 
\tensor{\Pi'}=\tensor{\Pi}\times\vec{a}=\vec{i}_1(\vec{p}_1\times\vec{a})+\vec{i}_2(\vec{p}_2\times\vec{a})+\vec{i}_3(\vec{p}_3\times\vec{a}).
\]
}
\end{dfn} \pause 

\begin{dfn}
\parbox{\textwidth}{
Под \alert{векторным произведением вектора} $\vec{a}$ \alert{на тензор} $ \tensor{\Pi} $ \alert{слева} понимается новый тензор $ \tensor{\Pi''} $, вычисленный по формуле:
\[ 
\tensor{\Pi''}=\vec{a}\times\Pi=(\vec{a}\times\vec{i}_1)\vec{p}_1+(\vec{a}\times\vec{i}_2)\vec{p}_2+(\vec{a}\times\vec{i}_3)\vec{p}_3.
\]
}
\end{dfn} \pause 
}

\frame{
\frametitle{Векторное произведение диады на вектор}

\[ 
(\vec{b}\vec{c})\times\vec{a}= \pause 
(\basis{i}_1b_1\vec{c}+\basis{i}_2b_2\vec{c}+\basis{i}_3b_3\vec{c})\times\vec{a}= \pause 
\]
\[
=
\basis{i}_1b_1(\vec{c}\times\vec{a})+\basis{i}_2b_2(\vec{c}\times\vec{a})+\basis{i}_3b_3(\vec{c}\times\vec{a})= \pause 
\vec{b}(\vec{c}\times\vec{a}).
\] \pause 

\medskip
\[ 
\vec{a}\times(\vec{b}\vec{c})= \pause 
\vec{a}\times(\basis{i}_1b_1\vec{c}+\basis{i}_2b_2\vec{c}+\basis{i}_3b_3\vec{c})= \pause 
\]
\[
=
(a\times\basis{i}_1) (b_1 \vec{c}) + (a\times\basis{i}_2) (b_2 \vec{c}) + (a\times\basis{i}_3) (b_3 \vec{c})= \pause 
\]
\[
=
(a\times \basis{i}_1b_1 ) \vec{c} + (a\times\basis{i}_2b_2) \vec{c} + (a\times\basis{i}_3b_3) \vec{c} = \pause 
\]
\[
=
(a\times \basis{i}_1b_1 + a\times\basis{i}_2b_2 + a\times\basis{i}_3b_3) \vec{c} = \pause 
(a\times (\basis{i}_1b_1 + \basis{i}_2b_2 + \basis{i}_3b_3)) \vec{c}=
\] \pause 
\[
=
(\vec{a}\times\vec{b})\vec{c}.
\]
}


\frame{
\frametitle{Пример}
\parbox{\textwidth}{
 Рассмотрим единичный тензор $\tensor{I}=\basis{i}_1\basis{i}_1+\basis{i}_2\basis{i}_2+\basis{i}_3\basis{i}_3$. \pause 
Построим  тензор $ \tensor{\Psi} $ 
\[ 
\tensor{\Psi}=\vec{\omega}\times \tensor{I}=(\vec{\omega}\times\basis{i}_1)\basis{i}_1+(\vec{\omega}\times\basis{i}_2)\basis{i}_2+(\vec{\omega}\times\basis{i}_3)\basis{i}_3.
\] \pause 
Умножим тензор $\tensor{\Psi}$  на произвольный вектор $\vec{a}$ справа
\[ 
\tensor{\Psi}\cdot\vec{a}= \pause 
(\vec{\omega}\times\basis{i}_1)(\basis{i}_1\cdot\vec{a})+(\vec{\omega}\times\basis{i}_2)(\basis{i}_2\cdot\vec{a})+(\vec{\omega}\times\basis{i}_3)(\basis{i}_3\cdot\vec{a})= \pause 
\]
\[
=(\vec{\omega}\times\basis{i}_1)a_1+(\vec{\omega}\times\basis{i}_2)a_2+(\vec{\omega}\times\basis{i}_3)a_3= \pause 
\]
\[
=\vec{\omega}\times\vec{a}.
\] \pause 

Таким образом, любой антисимметричный тензор может быть представлен в виде
\[ 
\tensor{A}=\vec{\omega}\times \tensor{I}.
\]
}
}

\frame{
\frametitle{Произведение тензоров}
\parbox{\textwidth}{
Рассмотрим два тензора $\tensor{A}$ и $\tensor{B}$ и вектор $ \vec{c} $. Тогда пусть
\[ 
\vec{c'}=\tensor{B}\cdot\vec{c}.
\]
и 
\[
\vec{c''}=\tensor{A}\cdot\vec{c'}=\tensor{A}\cdot (\tensor{B}\cdot\vec{c}).
\]
} \pause 

\begin{dfn}
\parbox{\textwidth}{
Если переход от вектора $ \vec{c} $ к вектору $ \vec{c''} $ осуществляется с помощью одного тензора $ \tensor{\Pi} $ со скалярными элементами $p_{kl}$:
\[ 
\vec{c''}=\tensor{\Pi}\cdot\vec{c},
\]
то тензор $ \tensor{\Pi} $ называется \alert{скалярным произведением тензоров} $\tensor{A}$ и $ \tensor{B} $:
\[ 
\tensor{\Pi}=\tensor{A}\cdot \tensor{B}.
\]
}
\end{dfn}
}

\frame{
\frametitle{Покомпонентные формулы для скалярного произведения тензоров}
}

\frame{
\frametitle{Определитель тензора}
\begin{dfn}
\parbox{\textwidth}{
\alert{Определителем тензора} $\tensor{\Pi}$ называется определитель матрицы его компонент:
\[ 
D(\Pi)=
\begin{vmatrix}
p_{11} & p_{21}  & p_{13}  \\
p_{21} & p_{22}  & p_{23}  \\
p_{31} & p_{32}   & p_{33}  
\end{vmatrix}.
\]
}
\end{dfn} \pause 

\begin{exampleblock}{Определитель произведения тензоров}
\parbox{\textwidth}{
Т.к. тензоры перемножаются как матрицы, то
\[
D(\Pi)=D(A)D(B).
\]
}
\end{exampleblock}

}

\frame{
\frametitle{Скалярное произведение диад}
\begin{theorems}
\normalfont
\parbox{\textwidth}{
Пусть $\tensor{A}=\vec{p}\vec{q}$ и $\tensor{B}=\vec{r}\vec{s}$, тогда
\[ 
\tensor{\Pi}=(\vec{p}\vec{q})\cdot(\vec{r}\vec{s})=(\vec{q}\cdot\vec{r})\vec{p}\vec{s}.
\]
} \pause 

\begin{proof}
\parbox{\textwidth}{
Для произвольного вектора $\vec{x}$ рассмотрим
\[
\tensor{\Pi}\cdot\vec{x}=  \pause 
(\vec{p}\vec{q})\cdot((\vec{r}\vec{s})\cdot\vec{x})=(\vec{p}\vec{q})\cdot (\vec{r}(\vec{s}\cdot\vec{x}))= \pause 
\vec{p}(\vec{q}\cdot\vec{r})(\vec{s}\cdot\vec{x})=
\] \pause 
\[
=((\vec{q}\cdot\vec{r})\vec{p}\vec{s})\cdot\vec{x}.
\] \pause 
Таким образом,
\[
\tensor{\Pi}=(\vec{q}\cdot\vec{r})\vec{p}\vec{s}.
\]
}
\end{proof}

\end{theorems}

}

\end{document}

